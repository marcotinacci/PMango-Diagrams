\section*{Ora e sede}
Seduta iniziata alle ore 13:55 presso l'aula studio dell'istituto di Matematica
"U.Dini", Viale Morgagni, Firenze

\section*{Ordine del giorno}
\begin{enumerate}
\item Divisione compiti per la codifica
\item Norme generali per la codifica
\item Stili dei commenti
\item Comportamenti da seguire in caso di difficolt\`a
\end{enumerate}

\subsection*{Approvazione dell'ordine del giorno}
Approvato da tutti i membri.

\section*{Discussione}
\begin{enumerate}
\item A ciascun membro \`e stato assegnato una parte da codificare in proprio.
\item Per evitare conflitti e per praticit\`a non si lavora mai in pi\`u di una
persona per file. Ogni classe metodo e variabile deve essere nominato secondo 
lo stile Upper/Lower CamelCase.
\item In linea generale il codice deve essere commentato in tutte le sezioni 
ritenute pi\`u difficoltose utilizzando in questo caso un commento in linea ("//
commento"). Ogni metodo e classe deve essere commentato invece tramite i 
commenti multi-linea ("/*  */") specificando i parametri accettati il valore di
ritorno, una descrizione dell'algoritmo e cosa fa il metodo/classe in poche 
parole. \`E stato convenuto inoltre di evitare commenti per metodi banali e per
le variabili d'appoggio.
\item In caso di difficolt\`a di qualsiasi genere e/o per segnalare approcci
diversi da quelli descritti sul documento del disegno del sistema segnalare 
prima possibile con un messaggio sul gruppo o se ritenuto particolarmente 
urgente direttamente per telefono.
\end{enumerate}

\section*{Varie e eventuali}
Questi gli argomenti sviluppati non previsti dall'ordine:
Su ogni macchina per lo sviluppo software in ambiente Windows \`e stato
installato EasyPHP versione 5.3.0, ed \`e deciso di utilizzare Eclipse PHP
version con installato Tigris SVN.

\section*{Sommario}
Non ci sono state conclusioni degne di nota in quanto i punti elencati prima
sono abbastanza chiari.

\section*{Chiusura della riunione}
Seduta terminata alle ore 14:55