\section*{Ora e sede}
Seduta iniziata alle 4.09 p.m. presso un aula del secondo piano dell'istituto
di matematica ''U. Dini'', viale Morgagni, Firenze.

\section*{Ordine del giorno}
\begin{enumerate}
\item comunicazioni interne
\item Processo di sviluppo
\item Notifica delle prime norme
\item Interfacciamento con mango
\item Giorno per il ritrovo
\item Discussione relativa al primo incontro con il committente
\item Matrice di responsabilita'
\end{enumerate}

\subsection*{Approvazione dell'ordine del giorno}
Approvato da tutti i membri.

\section*{Discussione}
\begin{enumerate}
\item comunicazioni interne: e' stato deciso di controllare il gruppo su gmail
con frequenza giornaliera e rispondere a ogni messaggio.
\item Processo di sviluppo: scelto un modello di sviluppo iterativo, ancora da
definire con precisione.
\item Notifica delle prime norme: esposta una breve introduzione al subversion. 
Utilizziamo Eclipse come ambiente di sviluppo, scaricando la versione di eclipse
dedicata per php. Per il client subversion utilizziamo il plugin disponibile 
tra i server di eclipse.
\item Interfacciamento con mango: introduzione all'utilizzo di pmango lato 
risorsa. Le risorse si recano sul task che intendono
sviluppare e per dichiarare il lavoro svolto creare un nuovo log.
Il project manager cerchera di usare una granularita' fine per l'individuazione 
per i vari task.
\item Giorno per il ritrovo: non c'e giorno fisso, utilizzare skype per una chat 
in conferenza su giorni variabili (sabato migliore candidato). 
\item Discussione relativa al primo incontro con il committente: 
deciso per mercoledi di proporre agli altri gruppi la \textbf{cooperazione}. \\
Discussione sul quando iniziare il processo di analisi. \\
Deciso per mandare una mail al profe \emph{per il docente} riguardante:
\label{ver:quesito_coop_gruppi}
\begin{itemize}
\item fattorizzazione dell'analisi, fondere i requisiti o dividerli in moduli?
\item dividere parte della lezione per la comunicazione con gli altri fornitori
\item in caso di collaborazione l'analisi va presentata da tutti per tutti gli
argomenti, oppure devono essere analisi focalizzate a specifici contesti?? 
(oppure convergenti per usare il punto prima)
\item proporre una collaborazione parziale se esiste un gruppo non interessato 
alla cooperazione
\end{itemize}
\item Matrice di responsabilita': illustrata e commentata la matrice di 
responsabilita' da parte del responsabile Marco Tinacci.
\end{enumerate}

\section*{Varie e eventuali}
Questi gli argomenti sviluppati non previsti dall'ordine:
\begin{description}
\item[nomine] nomina di \textbf{vice responsabile} a Francesco,
nomina di \textbf{analista} a Manuele
\item[analisi dei requisiti] non partire con il processo di analisi 
finche non abbiamo avuto risposta ai quesiti espressi al punto
\ref{ver:quesito_coop_gruppi} della discussione.
\end{description}

\section*{Sommario}
Non ci sono state conclusioni degne di nota in quanto i punti elencati prima
sono abbastanza chiari.

\section*{Chiusura della riunione}
Seduta terminata alle 5.43 p.m..
