\section*{Ora e sede}
	15:30 Viale Morgagni

\section*{Invitati e presenti}
	Committente PMango e Fornitori

\section*{Ordine del giorno}
\begin{itemize}
	\item Chiarimenti generali sul progetto
	\item Presentazione ultime interfacce
\end{itemize}

\section*{Discussione}
\begin{itemize}
	\item Chiarimenti generali sul progetto: 
Durante la durata dell'incontro sono state sollevate e chiarite le seguenti puntualizzazioni:
\begin{itemize}
\item Chiarite alcune notazioni da utilizzare, in particolare quella relativa ai TaskBox. La notazione da utilizzare va ridotta prima alle sole iniziali dei nomi se non basta allora si usa "..."
Ad esempio (riferirsi alla fig 3 di pag 3, del documento presentato dal committente) 6/40PH, 2/22PH D... R..., 2/14 W... S... 
\item Le date di fine attivit� si devono visualizzare solo se � al 100%
altrimenti NA.
\item Le rappresentazioni grafiche vanno sempre disegnate senza eseguire controlli sulla forma, sar� l'utente a trovare gli errori e nel caso correggerli.
\item Se due Critical Path hanno la stessa lunghezza a priorit� di durata si visualizza quello ad impegno minore, altrimenti ci si basa su il costo oppure infine su quella che ha la data di fine pi� vicina alla terminazione del progetto.
\item Al committente non interessano scroll bar laterali per la visualizzazione di Gif ampi.
\item Pu� essere interessante visualizzare l'immagine a dimensioni reali in una nuova scheda.
\item Legare la generazione di una GIF o di un PDF alla sessione, in modo che finch� uno rimane collegato pu� sfogliare anche precedenti generazioni di file. 
\item Predisporre la possibilit� di eliminare un report dalla sessione della Progect Report.
\end{itemize}

\item Presentazione ultime interfacce:
Sono state presentate le ultime interfacce realative ai TaskNetworks.
Durante la durata dell'incontro un altro fornitore ha presentato una demo di interfaccia funzionante, che nascondeva a comando le opzioni di ogni tab. La funzione per quanto interessante e piacevole presenta l'inconveniente che non � agevole e questo � stato in qualche modo trasmesso dal Committente.

\end{itemize}

\section*{Varie e eventuali}
Domande del team:
\begin{itemize}
\item Come dividere un Gantt in pagine sul PDF? L'opzione pu� essere comoda, visualizzando su pi� pagine il Gantt dividendolo per righe.
\item Come visualizzare i nuovi Report? I report dovranno contenere nella loro descrizione gli attributi selezionati oltre alla data di creazione. Aggiungere anche possibilit� di cancellare report senza dover riavviare la sessione.
\item Come gestire le dipendenze no well formed?  Devono sempre essere visualizzabili, non fare controlli.
\end{itemize}

\section*{Chiusura della riunione}
Riunione chiusa alle 17.30
