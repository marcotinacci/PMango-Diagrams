\section{Scope}

Questo \`e il \emph{Master Plan} per il progetto \textbf{PMango}. Questo piano
considera solo gli elementi software relativi alle aggiunte/modifiche descritte
nel documento di analisi. \\ \\
L'obiettivo primario di questo piano di test \`e quello di assicurare che la
nuova versione di \textbf{PMango} offrir\`a lo stesso livello di informazioni e
dettagli reso disponibile della versione corrente e aggiunger\`a tutte quelle
informazioni necessarie per raggiungere gli obiettivi modellati dal processo di
analisi.
\\ \\
Il progetto avr\`a tre livelli di testing i cui dettagli verranno definiti
nella sezione \ref{subsec:strategy} e negli specifici \emph{level plan}. 
\\ \\ 
Il quanto temporale stimato per questo progetto \`e molto compatto, quindi
\emph{ogni} ritardo nella fase di progettazione, sviluppo, installazione e
verifica possono avere effetti significati sul deploy finale.

\section{Struttura del documento}
Il documento del piano delle prove viene diviso in tre macro parti:
\begin{description}
  \item[planning] vengono identificati gli aspetti necessari per
  arrivare alla definizione della specifica dei singoli test. Si identificano
  tutte le funzionalit\`a (sia \emph{technical side} che \emph{customer side}),
  le risorse necessarie per eseguire il processo di test (soprattutto software 
  in quanto non esistono procedure che richiedono grande potenza di elementi 
  hardware), la strategia con cui si vuole portare avanti il progetto, i
  criteri di \emph{success, failure} per valutare l'esecuzione di un test e la
  divisione in livelli logici (\emph{fasi}) dell'intero processo.
  \item[design specification] per ogni insieme di funzionalit\`a catturate
  nella parte \emph{use cases} del documento di analisi, si crea un capitolo di
  design nel quale si specifica quali use case si vogliono esercitare, si
  raffina la strategia espressa nella parte di \emph{planning} e si
  identificano l'insieme di documenti \emph{case} che effettivamente
  implementano i test necessari per esercitare gli use cases scelti.
  \item[case specification] ogni documento di questa parte mira ad esercitare 
  un \emph{singolo} use case (o funzionalit\`a): si identificano l'insieme di
  triple \emph{(input, environment, output)} che permettono di costruire un 
  modello della batteria di test che verr\`a applicata sullo use case.
  
  Nella prima fase di stesura, non avendo ancora la possibilit\`a di
  riferirci ai sorgenti e quindi testare effettivamente il comportamento della
  funzionalit\`a (non possiamo creare dei task di prova con criteri atti a
  testare il codice scritto), la componente \emph{input} di ogni tripla contiene
  solo la quantit\`a di task del \emph{testing project} che verranno creati.
  
  Quando inizieremo la parte di codifica, la componente \emph{input} conterr\`a
  l'identificatore dei tasks che sono stati creati nel \emph{testing project}
  per esercitare la tripla.
\end{description}
Possiamo vedere i livelli sopra descritti come un test a tutti gli effetti, e
attribuire ad ognuno di essi un risultato, codificato in base ai
\emph{pass/fail criteria}. Per dare ogni singola valutazione, possiamo usare il
concetto della composizione di documenti in modo da soddisfare queste
relazioni $$panning \sqsubset desing_{specification} \sqsubset case_{
specification}$$
La relazione $\sqsubset$ definisce che nella coppia $a \sqsubset b$ la
valutazione di $a$ dipende da alcune valutazioni di $b$. Nella nostra catena 
abbiamo che il risultato atomico (non composto) dell'esecuzione di un test 
viene espresso dai relativi documenti \emph{case}, mentre gli altri documenti 
si limitano a comporre questi risultati per esprimere a loro volta il loro 
risultato (il concetto \`e simile al \emph{bubble up} dei risultati dai case 
dettagliati per tuple fino al piano totale).

\section{Elementi software sottoposti a test}
La seguente lista comprende oggetti software \emph{technical side}.
\begin{itemize}
  \item la classe \emph{ChartGenerator} e le sue specializzazioni, pacch
 etto \emph{Chart}, definite nella sezione \textbf{1.2} del documento
 \emph{disegno del sistema}
  \item la classe \emph{Commons} e le sue specializzazioni, pacchetto 
  \emph{commons}, definite nella sezione \textbf{1.1} del documento 
  \emph{disegno del sistema}
\end{itemize}

\section{Funzionalit\`a che verranno testate}
La seguente lista comprende oggetti software \emph{customer side}.
\begin{itemize}
  \item processo per la generazione di \emph{Gantt chart}
  \item processo per la generazione di \emph{WBS chart}
  \item processo per la generazione di \emph{Task Network chart}
  \item interfaccia grafica per la selezione delle nuove \emph{UserOption} per
  ogni \emph{Chart}
  \item aggiunta di ogni \emph{Chart} alla sezione di reportistica
\end{itemize}

\section{Software Risk Issues}
Ci sono alcuni punti che ci portano a definire questa sezione:
\begin{itemize}
  \item Reverse engineering di codice sorgente esistente, documentato nei sorgenti
ma non ha documenti ufficiali (apparte le tesi) che descrivino in modo chiaro
la struttura statica e dinamica di tutto il lavoro esistente.
  \item Uso di librerie esterne per la generazione delle immagini e dei
  documenti pdf.
\end{itemize}

\section{Strategia}

\subsection{Testing machines}
Identifichiamo due classi di macchine sulle quali viene condotto il processo di
testing:
\begin{itemize}
  \item \emph{develop machine} macchine dove avviene sia lo sviluppo del codice
  che parte del testing.
  \item \emph{acceptance machine} macchine (molto probabilmente unica, a meno
  di guasti) dove avviene solo il processo di testing.
\end{itemize}

\subsection{Testing projects}
\begin{description}
\item[testing project] Costruiamo un istanza di progetto contenente un insieme
di tasks, con le relative dipendenze sia di composizione che di ``finish to start'', necessari per
esercitare ogni tripla definita nella parte \ref{part:CaseSpecification} di 
ogni \emph{case specification}.
\\ \\
Questo progetto conterr\`a tasks costruiti in modo da catturare la maggior
parte dei contesti che potrebbero verificarsi nell'utilizzo a regime
dell'applicazione:
\begin{itemize}
  \item task con informazioni coerenti e ben formate, per avere un contesto in
  cui le informazioni appartengono al dominio e dovrebbero avere un output
  corretto
  \item task con informazioni non coerenti (avendo differenza fra le
  \emph{planned, actual} data per esempio), per avere un contesto in cui
  l'applicazione dovr\`a generare la notazione definita per i contesti anomali
  \item relazioni tra task ben formate (sia di composizione che di dipendenza)
  \item relazioni tra task non ben formate, in modo da verificare la bont\`a
  dell'adattamento dell'algoritmi di generazione per i casi anomali
\end{itemize} 
Questo progetto viene creato nella \emph{acceptance
machine} e su questa macchina sar\`a presente la versione completa del progetto necessaria per
soddisfare tutte le triple. 
\\ \\
Nelle \emph{develop machine} invece non \`e
richiesto necessariamente che sia presente l'istanza completa del progetto,
bensi sar\`a sufficiente riportare solo un sottoinsieme del progetto
sufficiente per testare il case che si vuole esercitare.
\\ \\
La struttura che possiamo dare al progetto al primo livello di dettaglio
(quindi immediatamente successivo la \emph{root}), \`e di creare una macro
attivit\`a relativa ad ogni \emph{case specification}, in modo che come sotto
attivit\`a possiamo creare i task nella configurazione richiesta dalla sezione
\emph{input} di ogni tripla. In questo modo otteniamo una struttura
che dedica un ramo di attivit\`a per ogni \emph{case specification}, rendendola
indipendente dalle altre. Allo stesso tempo triple di \emph{case specification}
 diverse possono comunque esercitare le loro triple usando rami dedicati ad
 altre \emph{case specification} (questo potrebbe essere un test di
 \emph{robustness}).
 
 \item[acceptance project] Questo progetto permette di avere una istanza di
 progetto che si avvicina pi\'u ad una istanza che potrebbe essere creata nella
 realt\'a, differenziandosi dalla precedente istanza di test soprattutto nella
 quantit\'a di informazioni che costituiscono il progetto. 
 
 Per creare questo progetto possiamo utilizzare la nostra istanza di
 \emph{Project Plan}, creando un backup e caricandolo sulla nostra macchina di
 \emph{acceptance}. Appena la fase di implementazione fornisce insiemi di
 elementi sufficienti per esercitare test su questo progetto, eseguiamo il
 backup del nostro \emph{project plan}. Quindi come limite inferiore per questa
 data, possiamo assumere la data odierna 15/12/2009.
 
 \end{description}
 
\subsection{Strumenti}
\label{subsec:testingTools}
Durante il processo di testing usufruiamo dei seguenti strumenti:
\begin{itemize}
  \item controllo visivo umano per quanto riguarda il confronto dell'output
  con l'output atteso (descritto nei documenti \emph{Test case specification})
  per quanto riguarda immagini
  \item utilizzo di verifica automatica di batterie di test usando l'insieme di
  oggetti contenuti nella distribuzione di \emph{phpUnit} oppure attraverso
  classi di test non appartenenti al precedente framework ma che hanno la
  stessa idea di verifica (controllo automatico tra output e risultato
  previsto).
\end{itemize}

\subsection{Test failure's metrics}
Modelliamo il concetto di \emph{failure} aggiungendo delle informazioni, in
modo da specializzarlo per il nostro processo di testing. Ogni
specializzazione ha il significato di esprimere la gravit\`a (importanza) del
fallimento. Quindi, l'esito di un test pu\`o essere \emph{success} o
\emph{failure}, nel secondo caso aggiungiamo queste specializzazioni:
\begin{description}
  \item[minor] il fallimento del test non \`e da considerarsi un evento grave.\\
  Gli oggetti software che hanno prodotto questa failure possono essere comunque
  inseriti nella release di \textbf{PMango 3}, non impediscono l'avanzare dello
  sviluppo. Possiamo identificare di questa failure come un \emph{defect}.
  \item[critical] il fallimento del test \`e da considerarsi un evento grave.\\
  Gli oggetti software che hanno prodotto questa failure non possono essere
  inseriti nella release di \textbf{PMango 3}, necessitano di ricontrollare il
  codice relativo a tali oggetti; non impediscono l'avanzare dello sviluppo.
  \item[blocking] il fallimento del test \`e da considerarsi un evento grave.\\
  Gli oggetti software che hanno prodotto questa failure non possono essere
  inseriti nella release di \textbf{PMango 3}, necessitano di ricontrollare il
  codice relativo a tali oggetti e, se necessario, ricontrollare il relativo
  documento di progettazione; impediscono l'avanzare dello sviluppo.
\end{description}
Queste misure sono valide per tutte le failure di test appartenenti a ogni
level plan descritto in \ref{sec:levelPlans}.


\section{Level plans}
\label{sec:levelPlans}
\subsection{Definitions}
\label{subsec:strategy}
Il processo di testing per il progetto \textbf{PMango} consiste nei livelli
seguenti.
\begin{description} 
\item[unit] questo livello viene effettuato da tutti gli
sviluppatori e stilato dal team dei verificatori con un rapprensentante degli
sviluppatori. 
\\ \\
Ogni motivazione riguardo ogni singolo unit test deve essere resa disponibile e
documentata in modo chiaro o in un documento apposito\footnote{inserire un
riferimento al relativo documento di test case specification} oppure nel codice
nel caso che 
\begin{itemize}
	\item viene utilizzato un strumento automatico indicato nelle sezione
	\ref{subsec:testingTools}
	\item la motivazione ha una dimensione ragionevolmente corta che \`e
	possibile inserirla come commento nel codice 
\end{itemize}

Questo livello viene esercitato su macchine di tipo \emph{develop machine}.

\item[system/integration] questo livello viene eseguito dal
team dei verificatori in presenza di un rappresentante degli sviluppatori
se necessario. 
\\ \\
Ogni motivazione e descrizione di questi test deve essere esposta nei documenti
\emph{Test case specification}
\\ \\
Questo livello viene esercitato su macchine di tipo \emph{acceptance machine} e
\emph{develop machine}.

\item[acceptance] questo livello viene eseguito dal cliente in presenza di un
rapprensentante dei verificatori.
\\ \\
Una volta terminato il livello di \emph{acceptance} il prodotto viene
rilasciato al cliente il quale pu\`o continuare la fase di testing in parallelo
alla fase di utilizzo.
\\ \\
Questo livello viene esercitato su macchine di tipo \emph{develop machine}.

\end{description}

\subsection{Precedence Relation}
Possiamo costruire la relazione di precedenza $\sqsubset$ fra coppie di level
plan che permette ad un oggetto software di avanzare nel processo di testing,
in modo da garantire il suo corretto funzionamento durante la fase di
revisione congiunta oppure nel collaudo. 

Definiamo $\sqsubset$ in questo modo: 
\begin{description}
  \item[unit $\sqsubset$ system/integration] ogni oggetto software entra nel 
processo di testing dal level plan \emph{unit}. 

Quando soddisfa tutti i suoi \emph{unit} test oppure, per ogni fallimento, la
metrica misura \emph{minor}, allora pu\`o essere disponibile per il level plan 
\emph{system/integration} se \`e richiesto da qualche \emph{system/integration}
test.
  \item[system/integration $\sqsubset$ acceptance] quando un oggetto (o gruppo
  di oggetti) superano tutti i test oppure, per ogni fallimento, la metrica 
  misura \emph{minor}, allora l'oggetto (o gruppo di oggetti) pu\`o avanzare
  nel level plan \emph{acceptance}
\end{description}
La relazione $\sqsubset$ \`e riflessiva, in quanto un test permane in un level
plan finche non soddisfa i requisiti per passare nel successivo; non \`e ne
simmetrica ne transitiva, in quanto vogliamo garantire al committente la
sequenzialit\`a del processo di testing.

Osservazione: quando un test passa da un livello al successivo deve essere
continuo rispetto alla batteria dei test specificata nel livello che lascia.
Ovvero: se avanza di livello deve continuare a soddisfare le condizioni che gli
hanno permesso di avanzare fino al livello corrente.



\section{Pass/fail criteria}
Il risultato dell'intero piano delle prove \`e dato dalla seguente relazione:
\begin{table}[h!]
  \begin{center}
    \begin{tabular}{| l | l |}
    \hline
    \textbf{risultato} & \textbf{criteri} \\
	\hline    
	success & numero di \emph{minor} failures $\leq 10$   \\
    \hline
    \emph{minor} failure & numero di \emph{minor} failures $> 10$ \\
    \hline
    \emph{critical} failure & almeno una \emph{critical} failure \\
    \hline
    \emph{blocking} failure & almeno una \emph{blocking} failure \\
    \hline
    \end{tabular}
  \end{center}
	\caption{La colonna \emph{criteri} si riferisce all'insieme dei \emph{design
	specification} definite in \ref{part:DesignSpecification}}
\end{table}

Il processo di testing verr\`a completato nella data in cui avverr\`a il
collaudo con il committente oppure quando questo piano da esito
\emph{success} in base alla relazione definita dalla tabella, prima del
collaudo. Dalla fine del processo di testing, la nuova versione di PMango
viene considerata \emph{live}. \\ \\ 
Nel caso in cui il nostro team non riuscisse a portare a termini gli impegni 
presi entro la data del collaudo, il processo di testing proseguir\`a fino alla
data in cui si considera terminato il tempo a disposizione per eseguire l'esame.

\section{Enviromental needs}
I seguenti elementi sono richiesti per supportare l'intero processo di testing:
\begin{itemize}
  \item Sia \emph{develop machine} che \emph{acceptance machine} devono avere
  installato una istanza di un server (L/W/M)AMP, con tutti i necessari permessi
  per la corretto funzionamento, relativamente al sistema operativo presente
  \item Sia \emph{develop machine} che \emph{acceptance machine} devono offrire
  tutte quelle \emph{third party resources} necessarie per l'utilizzo della
  nuova versione di PMango (fonts microsoft, \ldots) compresi tutte quelle
  necessarie per la versione di PMango attuale.
\end{itemize}
