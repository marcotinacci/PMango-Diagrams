\section{References}
Questi documenti sono riferiti all'interno del \emph{Master Plan}.
\begin{enumerate}
  \item Domain model \label{reference:domain_model}
  \item Use cases \label{reference:use_cases}
  \item Requisiti del prodotto  \label{reference:requirements}
\end{enumerate}

\section{Introduction}

Questo \`e il \emph{Master Plan} per il progetto \textbf{PMango}. Questo piano
considera solo gli elementi software relativi alle aggiunte/modifiche descritti
nei documenti \ref{reference:domain_model} e \ref{reference:requirements}.
\\ \\
L'obiettivo primario di questo piano di test \`e quello di assicurare che la
nuova versione di \textbf{PMango} offrir\`a lo stesso livello di informazioni e
dettagli reso disponibile della versione corrente e aggiunger\`a tutte quelle
informazioni necessarie per raggiungere gli obiettivi modellati dal processo di
analisi.
\\ \\
Il progetto avr\`a tre livelli di testing: 
\begin{itemize}
  \item unit
  \item system/integration
  \item acceptance
\end{itemize}
I dettagli di ogni livello verranno definiti nella sezione \ref{subsec:strategy}
e negli specifici \emph{level plan}. \\ \\
Il quanto temporale stimato per questo progetto \`e molto compatto, quindi
\emph{ogni} ritardo nella fase di progettazione, sviluppo, installazione e
verifica possono avere effetti significati sul deploy finale.

\section{Elementi software sottoposti a test}

\begin{itemize}
  \item some components produced by the detailed desing team
\end{itemize}

\section{Funzionalit\`a che verranno testate}
\begin{itemize}
  \item processo per la generazione di \emph{Gantt chart}
  \item processo per la generazione di \emph{WBS chart}
  \item processo per la generazione di \emph{Task Network chart}
  \item interfaccia grafica per la selezione delle nuove \emph{UserOption} per
  ogni \emph{Chart}
  \item aggiunta di ogni \emph{Chart} alla sezione di reportistica
\end{itemize}

\section{Software Risk Issues}
Ci sono alcuni punti che ci portano a definire questa sezione:
\begin{itemize}
  \item Reverse engineering di codice sorgente esistente, documentato nei sorgenti
ma non ha documenti ufficiali (apparte le tesi) che descrivino in modo chiaro
la struttura statica e dinamica di tutto il lavoro esistente.
  \item Uso di librerie esterne per la generazione delle immagini e dei
  documenti pdf.
\end{itemize}

\section{Strategia}

\subsection{Level plans}
\label{subsec:strategy}
Il processo di testing per il progetto \textbf{PMango} consiste nei livelli
seguenti. Abbiamo una relazione di sequenza espressa dall'ordine in cui vengono
descritti: l'idea principale \`e che l'insieme degli oggetti software parte dal
livello \emph{unit} e avanza verso i successivi in base ai predicati espressi
nelle seguenti descrizioni:
\begin{description} 
\item[unit] questo livello di testing viene effettuato da tutti gli
sviluppatori e stilato dal team dei verificatori con un rapprensentante degli
sviluppatori. 
\\ \\
Ogni motivazione riguardo ogni singolo unit test deve essere resa disponibile e
documentata in modo chiaro nella relativa specifica oppure nel codice se viene
utilizzato un tool automatico (phpunit??)
\\ \\
\`E il livello di partenza in cui ogni oggetto software che viene prodotto
appartiene. \emph{identificare i livelli di gravita' di ogni test??}
\\ \\
Si puo passare al livello successivo solo quando tutti i difetti critici sono
stati corretti.

\item[system/integration] questo livello di testing viene eseguito dal
team dei verificatori in presenza di un rappresentante degli sviluppatori
se necessario. No specific test tools are available for this project. 

% the following is important???
% A program may have up to two Major defects as long as they do not impede
 testing of the program (I.E. there is a work around for the error).
\\ \\
Si passa al livello successivo solo quando tutti i difetti critici e major sono
stati corretti

\item[acceptance] questo livello viene eseguito dal cliente in presenza di un
rapprensentante dei verificatori.
\\ \\
Una volta terminato il livello di \emph{acceptance} il prodotto viene
rilasciato al cliente il quale pu\`o continuare la fase di testing in parallelo
alla fase di utilizzo.

\end{description}

\subsection{Configuration Management}
L'insieme di oggetti software vengono testati sulle macchine con questa
configurazione:
\begin{itemize}
  \item tutti i test che rientrano nel \emph{unit level plan} vengono eseguiti
  in locale sulle macchine di sviluppo.
  \item quando gli oggetti software raggiungono una certa stabilit\`a e vengono
  soddisfatti i test di \emph{unit level}, i successivi test appartenenti al
  \emph{integration level} vengono eseguiti su una macchina dedicata (che
  sar\`a quella che verr\`a utilizzata nella seconda revisione congiunta)
\end{itemize}

\subsection{Test failure's metrics}
Possiamo associare ad ogni fallimento di un test un oggetto che ne esprime la
gravit\`a:
\begin{description}
  \item[minor] il fallimento del test non \`e da considerarsi un evento grave.
  Gli oggetti software che hanno prodotto questa failure possono essere cmq
  inseriti nella release di \textbf{PMango 3}, non impediscono l'avanzare dello
  sviluppo.
  \item[critical] il fallimento del test \`e da considerarsi un evento grave.
  Gli oggetti software che hanno prodotto questa failure non possono essere
  inseriti nella release di \textbf{PMango 3}, necessitano di ricontrollare il
  codice relativo a tali oggetti; non impediscono l'avanzare dello sviluppo.
  \item[blocking] il fallimento del test \`e da considerarsi un evento grave.
  Gli oggetti software che hanno prodotto questa failure non possono essere
  inseriti nella release di \textbf{PMango 3}, necessitano di ricontrollare il
  codice relativo a tali oggetti e, se necessario, ricontrollare il relativo
  documento di progettazione; impediscono l'avanzare dello sviluppo.
\end{description}

\section{Item pass/fail criteria}
Il processo di testing verr\`a completato nella data in cui avverr\`a il
collaudo con il committente nel caso i requisiti siano stati \emph{tutti}
implementati. Da questo data in poi la nuova versione di PMango viene
considerata \emph{live}. \\ \\
Nel caso in cui il nostro team non riuscisse a portare a termini
gli impegni presi entro la data del collaudo, il processo di testing
proseguir\`a fino alla data in cui si considera terminato il tempo a 
disposizione per eseguire l'esame.

\section{Enviromental needs}
I seguenti elementi sono richiesti per supportare l'intero processo di testing:
\begin{itemize}
  \item Sia \emph{develop machine} che \emph{acceptance machine} devono avere
  installato una istanza di un server LAMP, con tutti i necessari permessi per la corretto
  funzionamento, relativamente al sistema operativo presente
  \item Sia \emph{develop machine} che \emph{acceptance machine} devono offrire
  tutte quelle \emph{third party resources} necessarie per l'utilizzo della
  nuova versione di PMango (fonts microsoft, \ldots) compresi tutte quelle
  necessarie per la versione di PMango attuale.
\end{itemize}
