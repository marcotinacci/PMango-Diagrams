\section{References}
Questi documenti sono riferiti all'interno del \emph{Master Plan}.
\begin{enumerate}
  \item Domain model \label{reference:domain_model}
  \item Use cases \label{reference:use_cases}
  \item Requisiti del prodotto  \label{reference:requirements}
\end{enumerate}

\section{Introduction}

Questo \`e il \emph{Master Plan} per il progetto \textbf{PMango}. Questo piano
considera solo gli elementi software relativi alle aggiunte/modifiche descritti
nei documenti \ref{reference:domain_model} e \ref{reference:requirements}.
\\ \\
L'obiettivo primario di questo piano di test \`e quello di assicurare che la
nuova versione di \textbf{PMango} offrir\`a lo stesso livello di informazioni e
dettagli reso disponibile della versione corrente e aggiunger\`a tutte quelle
informazioni necessarie per raggiungere gli obiettivi modellati dal processo di
analisi.
\\ \\
Il progetto avr\`a tre livelli di testing: 
\begin{itemize}
  \item unit
  \item system/integration
  \item acceptance
\end{itemize}
I dettagli di ogni livello verranno definiti nella sezione \ref{sec:strategy} e
negli specifici \emph{level plan}. \\ \\
Il quanto temporale stimato per questo progetto \`e molto compatto, quindi
\emph{ogni} ritardo nella fase di progettazione, sviluppo, installazione e
verifica possono avere effetti significati sul deploy finale.