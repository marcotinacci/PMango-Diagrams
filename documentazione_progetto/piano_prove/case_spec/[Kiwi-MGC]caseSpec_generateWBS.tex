\chapter{Generate WBS Case}
\label{chap:generateWBS}

\begin{paragraph}{Level plan relation}
Le seguenti triple sono da considerarsi appartenenti ad un level plan in questo
modo:
\begin{description}
\item[unit] se si stanno esercitando le triple
\ref{sec:WBSTaskRepresentation},  \ref{sec:WBSDimensionRepresentation} sul
\emph{testing project}, allora appartengono al \emph{unit level};  la
componente su cui viene eseguito lo unit testing \`e la classe  \emph{GifArea}
con le sue specializzazioni
  \item[system/integration] tutte le triple possono considerarsi incluse al
  \emph{system/integration level} in quanto i dati su cui vengono costruite le
  rappresentazioni vengono recuperate secondo quanto descritto nel
  \emph{sequence diagram} nella sezione \textbf{1.1 Commons, 1.3 Task Data
  Tree}  del documento \emph{disegno del sistema}.
  \item[acceptance] se si stanno esercitando le triple
  \ref{sec:WBSTaskRepresentation}, \ref{sec:WBSDimensionRepresentation} sul 
  \emph{acceptance testing project} allora appartengono al  \emph{acceptance
  level}. Questo perch\`e nel \emph{acceptance testing project} ci possono 
  essere task candidati per esercitare una tripla, per\`o inseriti in un 
  contesto pi\`u grande.
\end{description}
\end{paragraph}

\section{Task representation}
\label{sec:WBSTaskRepresentation}
Testare la rappresentazione grafica di un task.
\begin{description}
\item[input] 
\quad
\begin{itemize}
  \item task atomico, non suddiviso in sottotask. Per questa configurazione
  almeno 3 esempi
  \item task composto, visualizzando i sotto task del livello successivo. Per
  questa configurazione almeno 2 esempi
  \item task composto, collassando i sotto task.  Per questa configurazione
  almeno 2 esempi
  \item task con il contesto \emph{actual} coerente con il contesto
  \emph{planned}, almeno 2 esempi
  \item task con il contesto \emph{actual} non coerente con il contesto
  \emph{planned}, sia in modo lieve (``good news'') sia in modo grave (``bad
  news''), almeno 2 esempi per entrambe i tipi di incoerenza
\end{itemize}
\item[environment] le user option
\emph{ActualTimeFrameOption, CompletitionBarOption, CompletitionBarOption, 
ActualDataOption, AlertMarkUserOption, ResourcesDetailsOption, TaskNameOption}
determinano i fields che verranno mostrati nella rappresentazione grafica del task.
\item[output] come descritto nella sezione 2. del documento di specifica.
\end{description}

\section{Dimension of Task representation}
\label{sec:WBSDimensionRepresentation}
Testare la dimensione della rappresentazione grafica di un task.
\begin{description}
\item[input] 
\quad
\begin{itemize}
  \item task aventi un nome la cui larghezza entra nella dimensione specificata
  nel documento di specifica, almeno 3 esempi
  \item task aventi un nome la cui larghezza supera la dimensione specificata
  nel documento di specifica, almeno 3 esempi
  \item task aventi un numero di resources diverso per qualche coppia di task,
  almeno 5 esempi
\end{itemize}
\item[environment] valorizzazione della user option
\emph{ResourcesDetailsOption, TaskNameOption}, considerando la generazione sia
mostrando i nomi dei task, sia non mostrandoli.
\item[output] come descritto nella sezione 2
\end{description}

\section{Composition relations representation}
\label{sec:wbs_compositionRelationsRepresentation}
Testare la rappresentazione grafica delle dipendenze di composizione espresse
nel \emph{project plan}.
\begin{description}
\item[input]  \quad
\begin{itemize}
  \item almeno 10 task legati da relazioni di composizione
\end{itemize}
\item[environment] valorizzazione della user option
\emph{WBSTreeSpecification} con una delle sue due specializzazioni per
esprimere il livello di dettaglio, pi\`u altre opzioni per avere la
rappresentazione delle informazioni volute.
\item[output] come descritto nella sezione 2. e nella figura 4. del
documento di specifica.
\end{description}

\section*{Pass/fail criteria}
\label{sec:wbsCasePassFailCriteria}
La colonna \emph{criteri} si riferisce alla rappresentazione grafica prodotta 
dall'implementazione. 
\begin{table}[h!]
  \begin{center}
    \begin{tabular}{| l | p{100mm} |}
    \hline
    \textbf{risultato} & \textbf{criteri} \\
	\hline    
	success & 
    \begin{itemize}
  \item la costruzione della rappresentazione grafica \`e coerente con le
  notazioni espresse nel documento di specifica. Il posizionamento del 
 simbolo \emph{delta} (con la sua estensione negativa ``!'') pu\`o essere
 fissato dall'implementazione in quanto non vengono specificate dimensioni o 
 gap predefiniti
 \item l'ordine con cui vengono mostrati i campi della \emph{TaskNodeBox} non
 \`e significativo in quanto non esiste una sua definizione nel documento di
 specifica. Ogni soluzione proposta dall'implementazione \`e corretta.
\end{itemize}
\\
    \hline
    \emph{critical} failure & 
    \begin{itemize}
    \item i field che compongono la \emph{Taskbox} sono staccati fra di loro.
    \item la spaziatura, i punti di stacco dei segmenti rappresentanti le
    relazioni di composizione e i punti di uscita/ingresso di tali segmenti n
   elle \emph{Taskbox} non rispettano i requisiti espressi nella sezione
   \textbf{3} del documento di specifica
    \item non vengono rispettate le misure relative all gap minimo espresse
    nella sezione \textbf{2.2.2} del documento delle metriche \end{itemize}\\
    \hline
    \emph{blocking} failure & La costruzione delle componenti del diagramma
    non rispetta i vincoli definiti nella sezione \textbf{2} del documento di 
   specifica rilasciato dal committente. \\
    \hline
    \end{tabular}
  \end{center}
\end{table}