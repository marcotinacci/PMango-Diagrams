\chapter{Generate WBS Case}
\label{chap:generateWBS}

\section{Task representation}
Testare la rappresentazione grafica di un task.
\begin{description}
\item[input] 
\quad
\begin{itemize}
  \item task atomico, non suddiviso in sottotask. Per questa configurazione
  almeno 3 esempi
  \item task composto, visualizzando i sotto task del livello successivo. Per
  questa configurazione almeno 2 esempi
  \item task composto, collassando i sotto task.  Per questa configurazione
  almeno 2 esempi
  \item task con il contesto \emph{actual} coerente con il contesto
  \emph{planned}, almeno 2 esempi
  \item task con il contesto \emph{actual} non coerente con il contesto
  \emph{planned}, sia in modo lieve (``good news'') sia in modo grave (``bad
  news''), almeno 2 esempi per entrambe i tipi di incoerenza
\end{itemize}
\item[environment] le user option
\emph{ActualTimeFrameOption, CompletitionBarOption, CompletitionBarOption, 
ActualDataOption, AlertMarkUserOption, ResourcesDetailsOption, TaskNameOption}
determinano i fields che verranno mostrati nella rappresentazione grafica del task.
\item[output] come descritto nella sezione 2. del documento di specifica.
\end{description}

\section{Dimension of Task representation}
Testare la dimensione della rappresentazione grafica di un task.
\begin{description}
\item[input] 
\quad
\begin{itemize}
  \item task aventi un nome la cui larghezza entra nella dimensione specificata
  nel documento di specifica, almeno 3 esempi
  \item task aventi un nome la cui larghezza supera la dimensione specificata
  nel documento di specifica, almeno 3 esempi
  \item task aventi un numero di resources diverso per qualche coppia di task,
  almeno 5 esempi
\end{itemize}
\item[environment] valorizzazione della user option
\emph{ResourcesDetailsOption, TaskNameOption}, considerando la generazione sia
mostrando i nomi dei task, sia non mostrandoli.
\item[output] come descritto nella sezione 2
\end{description}

\section{Composition relations representation}
Testare la rappresentazione grafica delle dipendenze di composizione espresse
nel \emph{project plan}.
\begin{description}
\item[input]  \quad
\begin{itemize}
  \item almeno 10 task legati da relazioni di composizione
\end{itemize}
\item[environment] valorizzazione della user option
\emph{WBSTreeSpecification} con una delle sue due specializzazioni per
esprimere il livello di dettaglio, pi\`u altre opzioni per avere la
rappresentazione delle informazioni volute.
\item[output] come descritto nella sezione 2. e nella figura 4. del
documento di specifica.
\end{description}

\section*{Pass/fail criteria}
See the figure \ref{table:passfailCriteriaGanttGeneration}.
\begin{table}[h!]
  \begin{center}
    \begin{tabular}{| l | p{100mm} |}
    \hline
    \textbf{risultato} & \textbf{criteri} \\
	\hline    
	success & 
    \begin{itemize}
  \item l'output dell'implementazione \`e sovrapponibile al modello a meno 
di una tolleranza pari a circa \textbf{1mm}
\item I punti di stacco dei segmenti
delle dipendenze possono disposti in maniera diversa da quelli esposti nel test
\item I font possono essere di una famiglia e dimensione diversa in base anche alle
UserOption scelte dall'utente.
\item Le linee verticali rappresentanti un elemento appartenente alla
\emph{TimeGrain} possono essere rappresentate (oppure non comparire affatto) in modo diverso.
\item Tutte quelle linee che sono errori di imprecisione nella stesura del mockup non
sono da considerarsi valide per il confronto (si lascia al buon senso capire
quando un piccola sbavatura non \`e importante\ldots)      
\end{itemize}
\\
    \hline
    \emph{minor} failure & costruzione delle testata contenente le informazioni
    sulla grana temporale non coincide a meno di una telleranza pari a circa 
    \textbf{1mm} con quella descritta nel mockup
    \\
    \hline
    \emph{critical} failure & 
    \begin{itemize}
    \item i blocchi rappresentanti le informazioni \emph{planned, actual} dei
    tasks sono staccati fra di loro. 
    \item non vengono rispettate le misure relative alla grana temporale
    scelta espresse nella sezione \textbf{2.2.1} del documento delle metriche 
    \end{itemize}\\
    \hline
    \emph{blocking} failure & La costruzione delle componenti del diagramma
    non rispetta i vincoli definiti nella sezione \textbf{1} del documento di 
   specifica rilasciato dal committente. \\
    \hline
    \end{tabular}
  \end{center}
	\caption{La colonna \emph{criteri} si riferisce alla rappresentazione grafica
	prodotta dall'implementazione}
	\label{table:passfailCriteriaGanttGeneration}
\end{table}