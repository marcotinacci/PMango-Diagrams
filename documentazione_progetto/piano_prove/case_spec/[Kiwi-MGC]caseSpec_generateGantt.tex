\chapter{Generate Gantt Case}
\label{chap:generateGantt}

% the following lines must move to the structure of the document
\footnote{put this paragraph within a section inside a test plan
introduction}Ogni successiva sezione rappresenta la tripla di testing
\emph{(Input, Environment, Output)}. Ogni tripla ha l'obiettivo di identificare un oggetto del diagramma che si vuole testare. Quando la componente \emph{Environment} non
viene specificata significa che la funzionalit\`a non richiede condizioni
particolari per il suo avvenimento (ad esempio le dipendenze vengono mostrate
solo se l'utente seleziona \emph{ShowDependenciesUserOption}).

\section{Basic task representation}
Testare la rappresentazione grafica di un task atomico, non composto da sotto
task.
\begin{description}
\item[input] almeno 4 esempi
\item[environment] le \emph{UserOption} \emph{FromStartRange, ToEndRange}
devono definire un intervallo di tempo nel quale l'input \`e almeno compreso.
\item[output] come descritto nella sezione 1. del
documento di specifica.
\end{description}

\section{Composed task representation}
Testare la rappresentazione grafica di un task composto.
\begin{description}
\item[input] 
\quad
\begin{itemize}
  \item si vogliono visualizzare i sottotask di livelli successivo al task
  composto. Per questa configurazione almeno 2 esempi
  \item si vuole una rappresentazione collassata dei sottotask del task
  composto. Per questa configurazione almeno 2 esempi
\end{itemize}
\item[environment] valorizzazione della user option
\emph{UserCustomSpecification = checked}, per permettere all'utente di
collassare alcuni task
\item[output] come descritto nella sezione 1
\end{description}

\section{Actual time representation}
Testare la rappresentazione grafica delle informazione del contesto
\emph{actual} di un task.
\begin{description}
\item[input]  \quad
\begin{itemize}
  \item un task terminato in base alle informazioni \emph{actual} prima della
  linea verticale che indica \emph{today}. Almeno 4 esempi
  \item un task non terminato in base alle informazioni \emph{actual} prima della
  linea verticale che indica \emph{today}. Almeno 4 esempi
\end{itemize}
\item[environment] valorizzazione della user option
\emph{UserCustomSpecification = checked}, per permettere all'utente di
collassare alcuni task
\item[output] come descritto nella sezione 1. e nella figura 2. del
documento di specifica.
\end{description}

\section{Task identifier and name representation}
Testare la rappresentazione grafica dell'identificativo e del nome di un task
sulla colonna di sinistra.
\begin{description}
\item[input]  \quad
\begin{itemize}
  \item il nome non supera i limiti espressi nel documento di specifica. Almeno
  2 esempi
  \item il nome supera i limiti espressi nel documento di specifica. Almeno 4 
  esempi
\end{itemize}
\item[environment] valorizzazione della user option
\emph{ShowTaskName}, per permettere all'utente mostrare oppure no il nome del
task
\item[output] come descritto nella sezione 1. del
documento di specifica.
\end{description}

\section{Task effort/resources representation}
Testare la rappresentazione grafica dell'effort e delle resources sulla destra
della \emph{Taskbox}.
\begin{description}
\item[input]  \quad
\begin{itemize}
  \item Supponiamo task composti. Almeno 3 esempi
  \item Supponiamo task base. Almeno 4 esempi
\end{itemize}
\item[environment] valorizzazione della user option \emph{ShowEffort}
\item[output] come descritto nella sezione 1. del
documento di specifica.
\end{description}

\section{Dependency relations representation}
See \ref{sec:TNDependencyrepresentation}.

\section*{Pass/fail criteria}
La colonna \emph{criteri} si riferisce alla rappresentazione grafica prodotta
dall'implementazione. 
\begin{table}[h!]
  \begin{center}
    \begin{tabular}{| l | p{100mm} |}
    \hline
    \textbf{risultato} & \textbf{criteri} \\
	\hline    
	success & 
    \begin{itemize}
  \item la costruzione della rappresentazione grafica \`e coerente con le
  notazioni espresse nel documento di specifica. Il posizionamento delle
  informazioni sull'\emph{effort} e sulle \emph{resources} pu\`o essere fissato
  dall'implementazione in quanto non vengono specificate dimensioni o gap
  predefiniti
  \item costruzione delle testata contenente le informazioni
    sulla grana temporale pu\`o avere una formattazione diversa da quanto
    mostrato nella sezione \emph{Mockups} del documento di analisi, basta che
    vengano mostrati i campi relativamente alla grana scelta (vedere relazione 
    descritta nella sezione \textbf{2.2.1} del documento  \emph{Requisiti del
    prodotto}).
\item Le linee verticali rappresentanti un elemento appartenente alla
\emph{TimeGrain} possono essere rappresentate (oppure non comparire affatto) in
modo diverso dall'implementazione attuale di PMango.      
\end{itemize}
\\
    \hline
    \emph{minor} failure & le \emph{Strip} che rappresentano le informazioni
    come descritte nel documento di specifica possono avere delle differenze
    con una tolleranza di \textbf{2mm} rispetto alla figura 2 del documento di
    specifica. (quindi anche riguardo alle differenti altezze)
    \\
    \hline
    \emph{critical} failure & 
    \begin{itemize}
     \item i punti di stacco dei segmenti e le spaziature
delle dipendenze possono essere fissati in maniera diversa da come viene
eseguito nella versione attuale di PMango
    \item i blocchi rappresentanti le informazioni \emph{planned, actual} dei
    tasks sono staccati fra di loro. 
    \item non vengono rispettate le misure relative alla grana temporale
    scelta espresse nella sezione \textbf{2.2.1} del documento delle metriche 
    \end{itemize}\\
    \hline
    \emph{blocking} failure & La costruzione delle componenti del diagramma
    non rispetta i vincoli definiti nella sezione \textbf{1} del documento di 
   specifica. \\
    \hline
    \end{tabular}
  \end{center}
	\label{table:passfailCriteriaGanttGeneration}
\end{table}