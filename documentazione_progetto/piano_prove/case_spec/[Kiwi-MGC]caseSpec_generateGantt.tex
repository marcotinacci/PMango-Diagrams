\chapter{Generate Gantt Case}
\label{chap:generateGantt}

% the following lines must move to the structure of the document
Ogni successiva sezione rappresenta la tripla di testing \emph{(Input,
Environment, Output)}. Ogni tripla ha l'obiettivo di identificare un oggetto del
diagramma che si vuole testare. Quando la componente \emph{Environment} non
viene specificata significa che la funzionalit\`a non richiede condizioni
particolari per il suo avvenimento (ad esempio le dipendenze vengono mostrate
solo se l'utente seleziona \emph{ShowDependenciesUserOption}).

\section{Basic task representation}
Testare la rappresentazione grafica di un task atomico, non composto da sotto
task.
\begin{description}
\item[input] almeno 4 esempi
\item[environment] le \emph{UserOption} \emph{FromStartRange, ToEndRange}
devono definire un intervallo di tempo nel quale l'input \`e almeno compreso.
\item[output] come descritto nella sezione 1. del
documento di specifica.
\end{description}

\section{Composed task representation}
Testare la rappresentazione grafica di un task composto.
\begin{description}
\item[input] 
\quad
\begin{itemize}
  \item si vogliono visualizzare i sottotask di livelli successivo al task
  composto. Per questa configurazione almeno 2 esempi
  \item si vuole una rappresentazione collassata dei sottotask del task
  composto. Per questa configurazione almeno 2 esempi
\end{itemize}
\item[environment] valorizzazione della user option
\emph{UserCustomSpecification = checked}, per permettere all'utente di
collassare alcuni task
\item[output] come descritto nella sezione 1
\end{description}

\section{Actual time representation}
Testare la rappresentazione grafica delle informazione del contesto
\emph{actual} di un task.
\begin{description}
\item[input]  \quad
\begin{itemize}
  \item un task terminato in base alle informazioni \emph{actual} prima della
  linea verticale che indica \emph{today}. Almeno 4 esempi
  \item un task non terminato in base alle informazioni \emph{actual} prima della
  linea verticale che indica \emph{today}. Almeno 4 esempi
\end{itemize}
\item[environment] valorizzazione della user option
\emph{UserCustomSpecification = checked}, per permettere all'utente di
collassare alcuni task
\item[output] come descritto nella sezione 1. e nella figura 2. del
documento di specifica.
\end{description}

\section{Task identifier and name representation}
Testare la rappresentazione grafica dell'identificativo e del nome di un task
sulla colonna di sinistra.
\begin{description}
\item[input]  \quad
\begin{itemize}
  \item il nome non supera i limiti espressi nel documento di specifica. Almeno
  2 esempi
  \item il nome supera i limiti espressi nel documento di specifica. Almeno 4 
  esempi
\end{itemize}
\item[environment] valorizzazione della user option
\emph{ShowTaskName}, per permettere all'utente mostrare oppure no il nome del
task
\item[output] come descritto nella sezione 1. del
documento di specifica.
\end{description}

\section{Task effort/resources representation}
Testare la rappresentazione grafica dell'effort e delle resources sulla destra
della \emph{Taskbox}.
\begin{description}
\item[input]  \quad
\begin{itemize}
  \item Supponiamo task composti. Almeno 3 esempi
  \item Supponiamo task base. Almeno 4 esempi
\end{itemize}
\item[environment] valorizzazione della user option \emph{ShowEffort}
\item[output] come descritto nella sezione 1. del
documento di specifica.
\end{description}


\section*{Pass/fail criteria}
See the figure \ref{table:passfailCriteriaGanttGeneration}.
\begin{table}[h!]
  \begin{center}
    \begin{tabular}{| l | p{100mm} |}
    \hline
    \textbf{risultato} & \textbf{criteri} \\
	\hline    
	success & 
    \begin{itemize}
  \item l'output dell'implementazione \`e sovrapponibile al modello a meno 
di una tolleranza pari a circa \textbf{1mm}
\item I punti di stacco dei segmenti
delle dipendenze possono disposti in maniera diversa da quelli esposti nel test
\item I font possono essere di una famiglia e dimensione diversa in base anche alle
UserOption scelte dall'utente.
\item Le linee verticali rappresentanti un elemento appartenente alla
\emph{TimeGrain} possono essere rappresentate (oppure non comparire affatto) in modo diverso.
\item Tutte quelle linee che sono errori di imprecisione nella stesura del mockup non
sono da considerarsi valide per il confronto (si lascia al buon senso capire
quando un piccola sbavatura non \`e importante\ldots)      
\end{itemize}
\\
    \hline
    \emph{minor} failure & costruzione delle testata contenente le informazioni
    sulla grana temporale non coincide a meno di una telleranza pari a circa 
    \textbf{1mm} con quella descritta nel mockup
    \\
    \hline
    \emph{critical} failure & 
    \begin{itemize}
    \item i blocchi rappresentanti le informazioni \emph{planned, actual} dei
    tasks sono staccati fra di loro. 
    \item non vengono rispettate le misure relative alla grana temporale
    scelta espresse nella sezione \textbf{2.2.1} del documento delle metriche 
    \end{itemize}\\
    \hline
    \emph{blocking} failure & La costruzione delle componenti del diagramma
    non rispetta i vincoli definiti nella sezione \textbf{1} del documento di 
   specifica rilasciato dal committente. \\
    \hline
    \end{tabular}
  \end{center}
	\caption{La colonna \emph{criteri} si riferisce alla rappresentazione grafica
	prodotta dall'implementazione}
	\label{table:passfailCriteriaGanttGeneration}
\end{table}