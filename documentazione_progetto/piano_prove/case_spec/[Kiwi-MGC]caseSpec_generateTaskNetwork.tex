\chapter{Generate Task Network Case}
\label{chap:generateTaskNetwork}

\section{Task representation}
See \ref{sec:WBSTaskRepresentation}.

\section{Dimension of Task representation}
See \ref{sec:WBSDimensionRepresentation}

\section{Dependency relations representation}
Testare la rappresentazione grafica delle dipendenze di dipendenza espresse
nel \emph{project plan}.
\begin{description}
\item[input]  \quad
\begin{itemize}
  \item almeno 10 task divisi in 
  \begin{itemize}
  \item atomici
  \item collapsed, non visualizzando i sotto task
  \item not collapsed, visualizzando i sotto task 
  \end{itemize} legati da relazioni di dipendenza ``finish to start''.
  \item almeno 5 tasks (\emph{A, B, C, D, E}) che appartengon alla
relazione ``finish to start'' $$\rightarrow = \lbrace (A, C), (B, C), \ldots
\rbrace$$   
  \item almeno 5 tasks (\emph{A, B, C, D, E}) che appartengon alla
relazione ``finish to start'' $$\rightarrow = \lbrace (A, B), (A, C), \ldots
\rbrace$$
\end{itemize}
\item[environment] si valorizzano queste user option
\emph{ReplicateArrowUserOption, \\CompleteDiagraUserOption,
ShowTimeGapUserOption} e i precedenti input vengono ripetuti per ogni
combinazione dei possibili valori possibili per l'ambiente (8 casi in totale)
\item[output] come descritto nella sezione 2. e nella figura 4. del
documento di specifica.
\end{description}

\section{Critical path representation}
Testare la rappresentazione grafica dei possibili critical path
nel \emph{project plan}.
\begin{description}
\item[input]  \quad
\begin{itemize}
  \item almeno 10 task che producono un solo critical path
  \item almeno 10 task che producono pi\`u di un critical path (provare con 3??)
\end{itemize}
\item[environment] si valorizzano queste user option
\emph{ShowCriticalPath}
\item[output] come descritto nella sezione 2. e nella figura 4. del
documento di specifica.
\end{description}

\section*{Pass/fail criteria}
See the figure \ref{table:passfailCriteriaGanttGeneration}.
\begin{table}[h!]
  \begin{center}
    \begin{tabular}{| l | p{100mm} |}
    \hline
    \textbf{risultato} & \textbf{criteri} \\
	\hline    
	success & 
    \begin{itemize}
  \item l'output dell'implementazione \`e sovrapponibile al modello a meno 
di una tolleranza pari a circa \textbf{1mm}
\item I punti di stacco dei segmenti
delle dipendenze possono disposti in maniera diversa da quelli esposti nel test
\item I font possono essere di una famiglia e dimensione diversa in base anche alle
UserOption scelte dall'utente.
\item Le linee verticali rappresentanti un elemento appartenente alla
\emph{TimeGrain} possono essere rappresentate (oppure non comparire affatto) in modo diverso.
\item Tutte quelle linee che sono errori di imprecisione nella stesura del mockup non
sono da considerarsi valide per il confronto (si lascia al buon senso capire
quando un piccola sbavatura non \`e importante\ldots)      
\end{itemize}
\\
    \hline
    \emph{minor} failure & costruzione delle testata contenente le informazioni
    sulla grana temporale non coincide a meno di una telleranza pari a circa 
    \textbf{1mm} con quella descritta nel mockup
    \\
    \hline
    \emph{critical} failure & 
    \begin{itemize}
    \item i blocchi rappresentanti le informazioni \emph{planned, actual} dei
    tasks sono staccati fra di loro. 
    \item non vengono rispettate le misure relative alla grana temporale
    scelta espresse nella sezione \textbf{2.2.1} del documento delle metriche 
    \end{itemize}\\
    \hline
    \emph{blocking} failure & La costruzione delle componenti del diagramma
    non rispetta i vincoli definiti nella sezione \textbf{1} del documento di 
   specifica rilasciato dal committente. \\
    \hline
    \end{tabular}
  \end{center}
	\caption{La colonna \emph{criteri} si riferisce alla rappresentazione grafica
	prodotta dall'implementazione}
	\label{table:passfailCriteriaGanttGeneration}
\end{table}