\chapter{Generate Task Network Case}
\label{chap:generateTaskNetwork}

\begin{paragraph}{Level plan relation}
Le seguenti triple sono da considerarsi appartenenti ad un level plan in questo
modo:
\begin{description}
\item[unit] se si stanno esercitando le triple
\ref{sec:TNTaskRepresentation},  \ref{sec:dimensionTaskRepresentation} sul
\emph{testing project}, allora appartengono al \emph{unit level};  la
componente su cui viene eseguito lo unit testing \`e la classe  \emph{GifArea}
con le sue specializzazioni
  \item[system/integration] tutte le triple possono considerarsi incluse al
  \emph{system/integration level} in quanto i dati su cui vengono costruite le
  rappresentazioni vengono recuperate secondo quanto descritto nel
  \emph{sequence diagram} nella sezione \textbf{1.1 Commons, 1.3 Task Data Tree}
  del documento \emph{disegno del sistema}.
  \item[acceptance] se si stanno esercitando le triple
  \ref{sec:TNTaskRepresentation},  \ref{sec:dimensionTaskRepresentation} sul 
  \emph{acceptance testing project} allora appartengono al  \emph{acceptance
  level}. Questo perch\`e nel \emph{acceptance testing project} ci possono 
  essere task candidati per esercitare una tripla, per\`o inseriti in un 
  contesto pi\`u grande.
\end{description}
\end{paragraph}

\section{Task representation}
\label{sec:TNTaskRepresentation}
See \ref{sec:WBSTaskRepresentation}.

\section{Dimension of Task representation}
\label{sec:dimensionTaskRepresentation}
See \ref{sec:WBSDimensionRepresentation}

\section{Dependency relations representation}
\label{sec:TNDependencyrepresentation}
Testare la rappresentazione grafica delle dipendenze di dipendenza espresse
nel \emph{project plan}.
\begin{description}
\item[input]  \quad
\begin{itemize}
  \item almeno 10 task divisi in (i seguenti sono tutti da intendersi come
  foglie dell'esplosione del \emph{project plan} scelta per la generazione,
  altrimenti non si potrebbero disegnare le dipendenze come espresso nella
  sezione 4 del documento di specifica):
  \begin{itemize}
  \item atomici
  \item collapsed, non visualizzando i sotto task
  \item not collapsed, visualizzando i sotto task 
  \end{itemize} legati da relazioni di dipendenza ``finish to start''.
  \item almeno 5 tasks (\emph{A, B, C, D, E}) che appartengon alla
relazione ``finish to start'' $$\rightarrow = \lbrace (A, C), (B, C), \ldots
\rbrace$$   
  \item almeno 5 tasks (\emph{A, B, C, D, E}) che appartengon alla
relazione ``finish to start'' $$\rightarrow = \lbrace (A, B), (A, C), \ldots
\rbrace$$
\end{itemize}
\item[environment] si valorizzano queste user option
\emph{ReplicateArrowUserOption, \\CompleteDiagraUserOption,
ShowTimeGapUserOption} e i precedenti input vengono ripetuti per ogni
combinazione dei possibili valori possibili per l'ambiente (8 casi in totale)
\item[output] come descritto nella sezione 2. e nella figura 4. del
documento di specifica.
\end{description}

\section{Critical path representation}
\label{sec:tn_criticalPathRepresentation}
Testare la rappresentazione grafica dei possibili critical path
nel \emph{project plan}.
\begin{description}
\item[input]  \quad
\begin{itemize}
  \item almeno 10 task che producono un solo critical path
  \item almeno 10 task che producono pi\`u di un critical path (provare con 3??)
\end{itemize}
\item[environment] si valorizzano queste user option
\emph{ShowCriticalPath}
\item[output] come descritto nella sezione 2. e nella figura 4. del
documento di specifica.
\end{description}

\section*{Pass/fail criteria}
See the figure \ref{table:passfailCriteriaGanttGeneration}.
\begin{table}[h!]
  \begin{center}
    \begin{tabular}{| l | p{100mm} |}
    \hline
    \textbf{risultato} & \textbf{criteri} \\
	\hline    
	success & 
    \begin{itemize}
  \item vedi la relativa sezione \emph{success} definita per il wbs case
  specification \ref{sec:wbsCasePassFailCriteria}
\item nel caso si vogliano disegnare i \emph{critical path} la tabella pu\`o
essere posizionata ovunque nello spazio disponibile, in quanto nel documento di
specifica non viene fissata una posizione precisa.
\item la formattazione e la posizione dell'informazione \emph{Time gap} pu\`o
variare, basta che sia vicina con una tolleranza di \textbf{3mm} al segmento a
cui \`e riferita.
\end{itemize}
\\
    \hline
    \emph{minor} failure & la posizione dei task che non hanno dipendenze con
    altri task, (ma vengono solo completati, necessaria la scelta dell'opzione
    \emph{CompleteDiagramUserOption}, associandoli alla \emph{start/end
    milestone}), possono essere piazzati seguendo l'euristica proposta
    dall'implementazione perch\`e nel documento di specifica non ne viene
    richiesta una disposizione particolare.
    \\
    \hline
    \emph{critical} failure & 
    \begin{itemize}
    \item i field che compongono la \emph{Taskbox} sono staccati fra di loro.
    \item la spaziatura, i punti di stacco dei segmenti rappresentanti le
    relazioni di dipendenza non rispettano i requisiti espressi nella sezione
   \textbf{4} del documento di specifica
    \item non vengono rispettate le misure relative all gap minimo espresse
    nella sezione \textbf{2.2.3} del documento delle metriche \end{itemize}\\
    \hline
    \emph{blocking} failure & La costruzione delle componenti del diagramma
    non rispetta i vincoli definiti nella sezione \textbf{2} del documento di 
   specifica rilasciato dal committente. \\
    \hline
    \end{tabular}
  \end{center}
	\caption{La colonna \emph{criteri} si riferisce alla rappresentazione grafica
	prodotta dall'implementazione}
	\label{table:passfailCriteriaGanttGeneration}
\end{table}