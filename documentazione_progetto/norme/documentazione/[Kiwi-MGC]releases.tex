\section{Stringa di release}
Le revisioni dei documenti versionati devono essere indicate con la dicitura \textbf{R.r} dove \emph{R} indica la major release mentre \emph{r} indica le minor release. Quando si parla di versione si intende quella appena specificata. Per risalire al numero di versione registrato nel server SVN si dovrà fare riferimento al diario di progetto.
La prima release è fissata come \textbf{0.0}.
Si ha una minor release di un prodotto quando si risolvono alcuni problemi significativi noti nell'ultima versione, e si scrive che migliorie si sono fatte nel messaggio di commit. Si ha una major release quando si risolvono tutti i problemi significativi noti.
Ogni documento unico o major release deve essere stampato e firmato da tutti i responsabili seguendo la traccia della matrice di responsabilità. Inoltre ogni documento deve essere identificato dalla data di redazione e quella di accettazione di ogni singolo responsabile. 
