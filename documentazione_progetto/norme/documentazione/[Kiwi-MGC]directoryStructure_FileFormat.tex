\section{Struttura delle directory e formato file}
Tutti i documenti risiedono in formato digitale (.tex) nella cartella \emph{"$/$documentazione\_progetto"} del server SVN del progetto (specificato negli strumenti usati). Definiamo la relazione fra tipi di modello e posizione nel file system:
\begin{description}
\item[diario] qui e' presente il file \emph{[Kiwi-MGC]diario.tex} che compone 
le informazioni divise nelle successive directory. Inoltre viene committato
anche il relativo \emph{[Kiwi-MGC]diario.pdf}, generato a partire dal file .tex,
 ogni volta che il file sorgente viene committato.
	\begin{description}
		\item[archivio]
		\item[consegne]
	\end{description}
\item[verbali] i file contenenti
	\begin{description}
		\item[riunioni\_committente] \quad
			\begin{description}
				\item[collaudo]
				\item[incontri\_analisi]
				\item[revisioni\_congiunte]	
			\end{description}
		\item[riunioni\_interne] \quad
			\begin{description}
				\item[collettive]
				\item[parziali]
			\end{description}
		\item[riunioni\_SQ]
	\end{description}
\item[norme] \quad
	\begin{description}
		\item[comunicazioni\_interne] \quad
		\item[documentazione]
		\item[strumenti]	
	\end{description}
\item[images]
\end{description}

\subsection{Sintassi valide esclusi i verbali}
I nomi dei file che non appartengono alle foglie della precedente struttura
devono rispettare la seguente sintassi: \\ 
$<$nome file$>$ ::= [$<$team$>$-$<$proj$>$]$<$file$>$.$<$ext$>$ \\
$<$file$>$ ::= organigramma | diario | norme | requisiti | offerta | disegno\_sistema | piano\_prove | prova\_$<$data$>$ \\
$<$ext$>$ ::= tex \\
$<$data$>$ ::= AAAAMMGG (anno su quattro cifre, mese e giorno su due cifre per
 far coincidere l'ordinsamento alfabetico con l'ordinamento cronologico) \\
$<$team$>$ ::= Kiwi\\
$<$proj$>$ ::= MGC

\subsection{Sintassi per i verbali}
nome della cartella: verbale\_$<$data$>$ \\
nomi per i file:\\
\quad[$<$team$>$-$<$proj$>$]verbale\_body\_$<$data$>$.tex\\
\quad[$<$team$>$-$<$proj$>$]verbale\_$<$data$>$.pdf \\
\quad[$<$team$>$-$<$proj$>$]verbale\_$<$data$>$.tex 
