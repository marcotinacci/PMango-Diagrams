\chapter*{Release \textbf{1.1}}

\chapter{Overall diagram}

\begin{figure}[h!] 
	\centering
	\includegraphics[width=1\textwidth]{../Milestone1-DomainModel/DomainModel.png}
	\caption{Overall UML diagram}
	\label{fig:overallDiagram} 
\end{figure}

Questo diagramma comprende tutti i concetti che abbiamo identificato durante la
prima iterazione del blocco di analisi. 

Nella figura abbiamo una visione di insieme che pu\`o essere utile a fini di
codifica e progettazione del piano delle prove. Finch\`e si rimane invece nella
sfera della progettazione (analisi inclusa) potrebbe produrre dei dubbi in
quanto propone molti concetti; mentre si sta cercando di raffinare le varie relazioni
secondo noi \`e necessaria una vista pi\`u in dettaglio di composizioni
di pochi concetti che sono legati tra loro, lasciando tutti gli altri ad una
loro commento separato.

Procediamo nel seguito del documento nella descrizione di piccole composizioni
in modo da chiarire i motivi per cui sono stati creati concetti e relazioni fra
essi.

\chapter{Aspects descriptions}
% from here every document which is included with the input command must have a
% dedicated section whitin his body
\section{Task}
\label{sec:task}

\begin{figure}[h!] 
	\centering
	\includegraphics[width=0.4\textwidth]{../TaskDetail.png}
	\caption{task and its relations}
	\label{fig:task} 
\end{figure}

Molto probabilmente il concetto di \emph{Task} esiste gia nell'attuale
versione di \textbf{PMango 2.2.0}. Quello che abbiamo pensato \`e di introdurre
un \emph{glue layer} che ci permette di non apportare modifiche al codice
esistente di mango, ma lavorare con uno strato di intermezzo per essere il meno
intrusivi possibile e poter portare avanti il lavoro dipendendo solo dalle
nostri oggetti, facendo il minor riferimento al codice gia esistente.

Vogliamo rendere trasparente il concetto che un \emph{Task} sia un attivit\`a
singola (non scomponibile in sottoattivit\`a) che una attivit\`a scomposta. 

Costruiamo la relazione $\rightarrow$ che lega questi due concetti:
\begin{itemize}
  \item \emph{BasicTask} $\rightarrow$ attivit\`a di base, non ulteriormente
  scomponibili
  \item \emph{ComposedTask} $\rightarrow$ attivit\`a che sono composte da sotto
  attivit\`a
\end{itemize}
In questo modo possiamo trattare questi due tipi di attivit\`a in modo
interscambiabile e del tutto trasparente. Usando l'astrazione \emph{Task} non
ci importa se abbiamo una attivit\`a base o composta, in quanto cosi le abbiamo
portate ad avere interfaccie compatibili.

\section{TaskBox}
\label{sec:taskbox}

\begin{figure}[h!] 
	\centering
	\includegraphics[width=0.5\textwidth]{../TaskBoxDetail.png}
	\caption{taskbox and its specializations}
	\label{fig:taskbox} 
\end{figure}

Il \emph{TaskBox} \`e la rappresentazione grafica di un
\emph{Task} (\autoref{fig:task}). Questo concetto astrae su queste
specializzazioni:
\begin{itemize}
\item \emph{GanttTaskBox} che ci permetter\`a di costruire la rappresentazione
in un \emph{GanttChart} conformi alle norme fissate nel documento di specifica.

\item \emph{NodeTaskBox} che ci permetter\`a di costruire la rappresentazione
in un \emph{WBSChart} e in \emph{TaskNetworkChart} alle norme fissate nel 
documento di specifica.
\end{itemize}

Abbiamo usato il principio di incapsulare il concetto che varia, modellando il
concetto astratto di \emph{TaskBox} per avere questi vantaggi:
\begin{itemize}
  \item non legare un \emph{Chart} specifico a una rappresentazione specifica
  \item aggiungere una nuova rappresentazione consiste nel modellarla e
  dichiarare che si tratta di una specializzazione di \emph{TaskBox}
  \item potremo cambiare a runtime il tipo di rappresentazione voluta nel
  disegno di un \emph{Chart}, magari inserire in un \emph{WBSChart} una
  rappresentazione pensata per i \emph{GanttChart}
\end{itemize}
\section{Strip}
\label{sec:strip}

\begin{figure}[h!] 
	\centering
	\includegraphics[width=0.6\textwidth]{../StripDetail.png}
	\caption{kinds of strips}
	\label{fig:strip} 
\end{figure}
\section{Chart}
\label{sec:chart}

\begin{figure}[h!] 
	\centering
	\includegraphics[width=0.6\textwidth]{../ChartDetail.png}
	\caption{chart and building blocks}
	\label{fig:chart} 
\end{figure}
\section{Dependency}
\label{sec:dependency}

\begin{figure}[h!] 
	\centering
	\includegraphics[width=0.5\textwidth]{../Milestone1-DomainModel/DependencyDetail.png}
	\caption{dependencies}
	\label{fig:dependencies} 
\end{figure}

\emph{Dependency} modella il tipo di dipendenze che possiamo rappresentare in
un \emph{Chart}. Per implementare la specifica abbiamo bisogno di incapsulare
queste varianti:\footnote{dire in quali Chart vengono utilizzate}
\begin{itemize}
  \item \emph{Finish-ToStartDependency}: siano $a, b$ due \emph{Task} tali
  che $b$ non pu\`o iniziare finch\'e $a$ non sia completato. Questa relazione
  \`e catturata da questa specializzazione.
  \item \emph{HierarchycalDependency}: siano $a, b_{i}$ con $i= 1,\ldots,n \in
  N$, \emph{Task}s tali che $a$ \`e scoposto in $b_{i}$ \emph{Task}. Questa
  relazione \`e catturata da questa specializzazione.
\end{itemize}
\section{UserOption}
\label{sec:userOption}

\begin{figure}[h!] 
	\centering
	\includegraphics[width=0.7\textwidth]{../UserOptionDetail.png}
	\caption{UserOptions and choice}
	\label{fig:userOption} 
\end{figure}

Quando un generico client (potrebbe essere sia una persona fisica che un
oggetto astratto) della nostra implementazione della specifica vuole generare
un \emph{Chart} pu\`o guidare la generazione decidendo alcuni fattori che sono
di suo interesse. Questi fattori vengono modellati dal concetto di
\emph{UserOption}.

Ogni \emph{Chart} espone una lista di \emph{UserOption} per dare al client la
possibilit\`a di esprimere quali informazioni guidare. Questa lista varia da
\emph{Chart} a \emph{Chart}\footnote{creare un reference dove vengono mappate
questa relazione: potrebbe essere un appendici di questo documento??}.

Il concetto di lista di \emph{UserOption} \`e catturato in
\emph{UserOptionsChoice}.

Procediamo per passi: nelle prossime subsection osserviamo due aspetti che
trattarli insieme potrebbe non essere sufficiente per esporli in modo chiaro.

\subsection{\emph{UserOption}'s Instances}
Questa \`e stata una decisione non molto facile da prendere. Il problema \`e
questo: nella specifica abbiamo che per ogni \emph{Chart} il committente ha
dichiarato quali \emph{UserOption} mostrare. Queste per\`o non rappresentano un
concetto che vogliamo catturare nel nostro modello, ma allo stesso tempo sono
\emph{istanze} (un insieme discreto quindi) di elementi che fissa il
committente.

Per questo motivo decidiamo di codificare questo insieme discreto in questo
documento e la successiva enumerazione \`e da considerarsi parte integrante del
diagramma inserito come figura.

Rappresentiamo il concetto espresso sopra indicando due descrizioni con questa
struttura di codifica:
\begin{itemize}
  \item \emph{istanze}, dove inseriamo tutte le possibili \emph{UserOption} che
  non possono essere ancora raffinate
  \item \emph{specializzazioni} dove inseriamo tutte le possibili
  specializzazioni di \emph{UserOption} che possono essere ancora raffinate,
  ripetendo in modo ricorsivo questa struttura di codifica
\end{itemize}

Le successive descrizioni sono relative al concetto di \emph{UserOption}:
\begin{description}
  \item[istanze]\footnote{inserire qui la il mapping sui vari Chart?}
  \emph{WBSExplosionLevelUserOption, ActualTimeFrameOption, CompletitionBarOption, \\ PlannedDataOption, ActualDataOption,
  AlertMarkUserOption, ReplicateArrowUserOption, FindCriticalPathUserOption,
  WBSUserSpecificationUserLevel, \\WBSUserSpecificationUserLevel,
  ResourcesDetailsOption, TaskNameOption, CompleteDiagramUserOptions}
  \item[specializzazioni] \quad
  \begin{itemize}
    \item \emph{WBSTreeSpecification}
    \begin{description}
  \item[istanze] \emph{LevelSpecification, UserCustomSpecification}
  \item[specializzazioni] nessuna
\end{description}

\item \emph{TimeGrainUserOption}
    \begin{description}
  \item[istanze] \emph{WeaklyGrain, MonthlyGrain}
  \item[specializzazioni] nessuna
\end{description}

\item \emph{ImageDimensionUserOption}
    \begin{description}
  \item[istanze] \emph{CustomDim, FitInWindowDim, OptionalDim, DefaultDim}
  \item[specializzazioni] nessuna
\end{description}

\item \emph{TimeRangeUserOption}
    \begin{description}
  \item[istanze] \emph{CustomRange, WholeProjectRange, FromStartRange, ToEndRange}
  \item[specializzazioni] nessuna
\end{description}

  \end{itemize}
\end{description}

\subsection{\emph{UserOptionsChoice}}
Questo concetto \`e, secondo la nostra analisi, molto importante in quanto ci
permette di astrarre dal client che richiede una generazione.

Il motivo per cui abbiamo introdotto questo concetto \`e di poter lavorare lato
server usando \emph{UserOptionsChoice} per controllare quali informazioni il
client vuole guidare. In questo modo non siamo vincolati ad accedere ai dati
inviati per \emph{POST, GET} dalla form HTML, ma possiamo direttamente guardare
in \emph{UserOptionsChoice}. Queste ci permette di disaccopiare il processo di
generazione della maschera di input di una pagina HTML. 

Se vogliamo utilizzare il processo di generazione (che comunque \`e
server side) scrivendo un programma client (GUI o da riga di comando) che
costruisce una HTML request ad hoc (dovremo definire una grammatica e
attribuire la semantica ai contesti, questo \`e necessario, non ch\`e scrivere 
un parser), usando \emph{UserOptionsChoice} e il suo disaccoppiamento ci sar\`a
possibile farlo. 

Una volta ricevuta la response possiamo maneggiare la pagina
inviata come una response HTML valida e usarla per i nostri obiettivi (possiamo
rihiedere l'immagina generata, o il file PDF generato, salvandolo in
locale, oppure visualizzando lo stesso con un browser, ma possiamo anche
inserire in un db oppure farci dei test sopra\ldots).

Dovremo quindi costruire un oggetto che si incarica di costruire
\emph{UserOptionsChoice} in base al tipo di richiesta ricevuta (da una pagina
html come \`e il caso di PMango, oppure una richiesta da un client indipendente
scritto in un qualche linguaggio). Una volta costruito l'insieme delle
\emph{UserOption} \`e possibile iniziare la generazione. Questo sar\`a delegato
alla fase di progettazione.

\section{ReportSection}
\label{sec:reportSection}

\begin{figure}[h!] 
	\centering
	\includegraphics[width=0.5\textwidth]{../Milestone1-DomainModel/img/ReportSectionDetail.png}
	\caption{report section}
	\label{fig:reportSection} 
\end{figure}

Abbiamo da implemetare un requisito che vuole la possibilit\`a di aggiungere
alla reportistica un determinato \emph{Chart} con le relative \emph{UserOption}
scelte dall'utente. Modelliamo quindi il concetto di \emph{ReportSection} per
realizzare questo requisito. Come si vede dalla figura, \emph{ReportSection}
associa \emph{Chart} e \emph{UserOptionChoice}. Utilizziamo direttamente la
lista delle scelte\footnote{che viene costruita lato server} in modo da non
doverla costruire nella funzionalit\`a di assemblamento del report.

