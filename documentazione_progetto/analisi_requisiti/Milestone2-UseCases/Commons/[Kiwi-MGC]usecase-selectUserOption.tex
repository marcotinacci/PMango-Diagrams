\section{Select User Option}
\label{seq:selectUserOption}
\subsection{Basic course}
Si assume che il client stia interagendo con il tab relativo al \emph{Chart}
che vuole generare, quindi lo use case \ref{seq:showProjectPage} \`e gi\`a
stato completato. \\\\
Il client vuole selezionare alcune \emph{UserOption} per guidare la
rappresentazione delle informazioni che verranno codificate nel \emph{Chart}.\\
Queste sono rappresentate tramite una form html, sotto forma di controlli
grafici, dipendenti dal tipo e dalla semantica della \emph{UserOption} che
rappresentano. \footnote{inserire qua il mappaggio tra il tipo di UserOption e
il controllo grafico che viene bindato.}
Il sistema non effettua alcuna azione in quanto il fatto di memorizzare le
scelte fino al ``submit'' viene tenuto dalla form html stessa.
