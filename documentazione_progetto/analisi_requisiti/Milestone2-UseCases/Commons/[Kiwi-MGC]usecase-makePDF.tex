\section{Make PDF}
\label{seq:commonsMakePDF}
\subsection{Basic course}
Si assume che il client stia interagendo con il tab relativo al \emph{Chart}
che vuole generare.

Il client fa click sul pulsante ``Make PDF''.\\
Il sistema esegue questi passi:
\begin{itemize}
  \item invoca lo use case \ref{seq:makeUserOptionsChoice} per costruirsi la
  \emph{UserOptionsChoice}.
  \item esegue una ricerca dei \emph{Task} che devono essere inclusi nel
  \emph{Chart}.  
  \item invoca la specializzazione di \ref{seq:generateChart} per la
  creazione del relativo \emph{Chart}
  \item costruisce un file pdf aggiungendo al suo interno la rappresentazione
  generata
  \item invia al client una pagina di risposta con una \textbf{icona} accanto ai
  due pulsanti della reportistica, per segnalare che il file PDF \`e
  disponibile.
\end{itemize}