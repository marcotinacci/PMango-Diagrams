\section{Make Hierarchycal Dependencies}
\label{seq:makeHierarchycalDependencies}
\subsection{Basic course}
Il client richiede la funzionalità di rappresentazione della relazione 
gerarchica di un task.

Il sistema esegue questi passi:
\begin{enumerate}
  \item in base alle informazioni in \emph{WBSTreeSpecification}, recupera i
  figli del task (non l’intera discendenza, solo quelli di livello successivo) e
  li dispone graficamente\footnote{OSS. Non tutte le opzioni, in questa
  situazione hanno senso (resources detail?, alert marks?)} al di sotto di esso (top-down)\footnote{effettivamente non e' che e' lui che li crea, dovrebbe
  solo aggiungerci le spezzate.}
  \item Il sistema collega il task ai suoi figli con linee spezzate cdns.
\end{enumerate}

\subsection{Alternative course}
\begin{description}
\item[WBSExplosionLevel = 0] se il livello di visualizzazione della WBSStructure
richiesto si limita alla root, crea il solo un \emph{NodeTaskbox} rappresentante
l’intero progetto.
\end{description}

