\section{Make Hierarchycal Dependencies}
\label{seq:makeHierarchycalDependencies}
\subsection{Basic course}
Il client richiede la funzionalità di rappresentazione della relazione 
gerarchica di un task.

Il sistema esegue questi passi:
\begin{enumerate}
  \item in base alle informazioni in \emph{WBSTreeSpecification}, recupera i
  figli del task (non l’intera discendenza, solo quelli di livello successivo) e
  li dispone graficamente al di sotto di esso.
  \item La distanze fra livelli e tra taskbox appartenenti allo stesso
  livello rispettano queste regole:
  \begin{itemize}
    \item la distanza \emph{inter-Taskbox} fra coppie di
    \emph{Taskbox} adiacenti, rappresentati sullo stesso livello, deve essere
    equa ed uguale per ogni coppia (questa regola si ripete in modo iterativo
    per ogni coppia appartenente ad un livello $l$)
	\item la distanza \emph{inter-Level} fra coppie di livelli contenenti le 
	rappresentazioni dei task padre e i suoi figli deve essere equa e uguale
	per ogni coppia di livelli (questa regola si applica ricorsivamente al livello
	dei figli, nel caso che almeno un figlio sia composto da subtask)
  \end{itemize} 
  
  \item Il sistema collega il task ai suoi figli con linee spezzate cdns.
\end{enumerate}

\subsection{Alternative course}
\begin{description}
\item[WBSExplosionLevel = 0] se il livello di visualizzazione della WBSStructure
richiesto si limita alla root, crea il solo un \emph{NodeTaskbox} rappresentante
l’intero progetto.
\end{description}

