\section{WBS User Options}
\label{seq:wbsUserOption}
\subsection{Basic course}
Questo use case estende (specializza) lo use case \ref{seq:showUserOptions}.
\\ \\
Il sistema costruisce le \emph{UserOption} relative al \emph{WBSChart} in un
unico passo, composto da queste azioni omogenee:
\begin{itemize}
  \item rappresenta \textbf{TaskNameUserOption} con una checkbox e associa una
  label descrittiva
  \item rappresenta \textbf{PlannedDataUserOption} con una checkbox e associa una
  label descrittiva
  \item rappresenta \textbf{PlannedTimeFrameUserOption} con una checkbox e associa una
  label descrittiva
  \item rappresenta \textbf{ResourcesUserOption} con una checkbox e associa una
  label descrittiva
  \item rappresenta \textbf{ActualTimeFrameUserOption} con una checkbox e associa una
  label descrittiva
  \item rappresenta \textbf{ActualDataUserOption} con una checkbox e associa una
  label descrittiva
  \item rappresenta \textbf{AlertMarkUserOption} con una checkbox e associa una
  label descrittiva
  \item rappresenta le istanze di \textbf{WBSTreeSpecification} come voci di una
  combobox. Quando viene scelta la voce \textbf{LevelSpecificationUserOption}, allora
  compare vicino alla combobox, un'altra combobox con i livelli disponibili
  (come quella che esiste nell'attuale versione di PMango)
  \item rappresenta le istanze di \textbf{ImageDimensionUserOption} come voci di
  una combobox. Si aggiungo ulteriori controlli in base alle seguenti scelte:
  \begin{description}
    \item[CustomDimUserOption] allora costruisci accanto alla combo, due
    textbox per specificare la coppia di interi rappresentate le dimensioni del
    \emph{Chart} desiderate.
  \end{description}
  \item rappresenta \textbf{OpenInNewWindowUserOption} con una checkbox e associa una
  label descrittiva
\end{itemize}