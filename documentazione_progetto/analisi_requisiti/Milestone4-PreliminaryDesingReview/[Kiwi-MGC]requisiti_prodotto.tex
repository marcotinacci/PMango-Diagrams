\chapter*{Release \textbf{1.2}}

\chapter{Requisiti obbligatori}

\section{Generali (2.1)}
Come requisiti fondamentali, \textbf{PMango 3.0} sar\`a visualizzabile e
usabile con le ultime versioni \emph{Internet Explorer 8} e \emph{Mozilla
Firefox 3.0}.

Le nostre modifiche e aggiunte saranno distribuite senza costi di licenza, in
quanto si tratta di estensioni di un progetto GPL

Ci assumiamo la responsabilit\`a di essere conformi ai punti \emph{d), e)}.

\section{Diagrammi WBS, Gantt e Task Network (2.2)}
\begin{itemize}
  \item[a)] implementato nello use case \textbf{\ref{seq:showProjectPage}}.
  \item[b)] implementato negli use case \textbf{\ref{seq:makeNodeTaskbox}} e in
  \textbf{\ref{seq:makeGanttTaskbox}}.
  \item[c)] implementato nello use case \textbf{\ref{seq:showUserOptions}}
\end{itemize}

\section{Diagrammi specifici (2.3)}
\begin{itemize}
  \item[a)] implementato nello use case \textbf{\ref{seq:showUserOptions}}
  \footnote{replace every ShowUserOption reference with the relative
  generalization}
  \item[b)] implementato nello use case
  \textbf{\ref{seq:createCriticalPathTable}}
  \item[c)] implementato nello use case \textbf{\ref{seq:showUserOptions}}
  \item[d)] implementato nello use case \textbf{\ref{seq:showUserOptions}} e
  descritto in modo dettagliato nella sezione
  \textbf{\ref{subsec:UserOptionInstances}} del documento
  \textbf{Domain Model}
  \item[e)] realizzato nello use case \textbf{\ref{seq:showUserOptions}}
\end{itemize}

\section{Generazione di immagini e doc (2.4)}
\begin{itemize}
  \item[a)] realizzato negli use case \textbf{\ref{seq:refreshChart},
  \ref{seq:commonsMakePDF}}
  \item[b)] realizzato nello use case \textbf{\ref{seq:addToReportUserAction}}
  \item[c)] realizzato nello use case \textbf{\ref{seq:showUserOptions}} e
  descritto in modo dettagliato nella sezione
  \textbf{\ref{subsec:UserOptionInstances}}  del documento \textbf{Domain Model}
  \item[d)] vedi punto \emph{a)}
  \item[e)] realizzato nello use case \textbf{\ref{seq:openInNewWindow}}
\end{itemize}

\section{Documentazione (2.5)}
Ci assumiamo la responsabilit\`a di essere coerenti a quanto richiesto nei
punti \emph{a), b), c)} nei momenti in cui verranno effettivamente implementati.

\chapter{Semplificazioni, Metriche}
\section{Semplificazioni e requisiti aggiuntivi}
\begin{itemize}
  \item[a)] il nostro gruppo \textbf{non} prevede lo sviluppo di requisiti
  aggiuntivi, preferendo implementare correttamente il processo di sviluppo 
  adottato per raggiungere i requisiti richiesti dal committente.
  \item[b)] abbiamo deciso di creare \textbf{solo} oggetti \textbf{gif} per
  usarli in modo interscambiabile sia nella visualizzazione da browser web, sia
  per aggiungerli in documenti PDF. Questo ci porta alcuni vantaggi:
  \begin{itemize}
    \item ci interfacciamo con una sola libreria, avendo cosi modo di capirne a
    fondo il comportamento e eventualmente aggiungere quelle funzionalit\`a di
    helper che potrebbero servirci, ma che attualmente non vengono fornite.
    \item ci riduce il carico di lavoro, questo non preclude che se arriviamo
    in anticipo con un prodotto finito e che rispetta la specifica richiesta,
    potremo proporre una integrazione dell'offerta sviluppando le
    funzionalit\`a native per la rappresentazione in PDF.
    \end{itemize}
\end{itemize}

\section{Metriche}

\subsection{Metrica sulla grana temporale di un Gantt}
Vogliamo fissare un limite minimo di informazioni che vengono rappresentate
nella stampa su \emph{carta} di un \emph{Gantt chart} in base alla grana
temporale scelta.

Facciamo queste assunzioni:
\begin{itemize}
  \item la stampa del \emph{Gantt chart} \`e \textbf{landscape}.
  \item \emph{PrintableArea} \`e l'area del foglio A4 su cui \`e effettivamente
  possibile stampare. In \emph{PrintableArea} non rientrano i quattro margini
  di stampa della stampante che viene utilizzata per la stampa.
  
  \item area stampabile $PrintableArea_{width}$ , $PrintableArea_{height}$:
  rappresentano le dimensioni dell'area stampabile del foglio, rispettivamente 
  larghezza e altezza, quindi considerando gia eliminati i margini di stampa.
  
  \item Per la larghezza della colonna di sinistra del diagramma che ospita gli
  id e i nomi dei vari \emph{Task} si suppone valga questa uguaglianza:
  \begin{displaymath}
  	left\_column_{width} = \frac{PrintableArea_{width}}{6}
  \end{displaymath}
  
   \item Si suppone che l'utente non abbia selezionato le \emph{UserOption} per
   avere informazioni sulle risorse sulla destra della \emph{GanttTaskbox}.

	\item Sia $GrainAvailable = \lbrace ora
	, giorno, settimana, mese, anno	\rbrace$. Definiamo la relazione $\sqsubset$
	come:
	\begin{displaymath}
	\sqsubset = \lbrace (a, b) \in GrainAvailable \times GrainAvailable : b \quad
	aggregates \quad a
	\rbrace
	\end{displaymath}
	l'operatore $b \quad aggregates \quad a$ esprime che $b$ \`e composto da alcuni
	$a$.
	
	Usando la precedente relazione vale:
	\begin{displaymath}
  	 ora \sqsubset giorno \sqsubset settimana \sqsubset mese \sqsubset anno
  	\end{displaymath}
\end{itemize}

Definiamo la metrica in base alle possibili grane temporali:
\begin{description}
\item[ora] \quad
\begin{itemize}
  \item fissiamo la dimensione del gap fra una linea verticale tratteggiata
e la successiva uguale a \textbf{3mm}.
  \item Si rappresentano \textbf{24} ore per giorno, per supportare progetti
  mission critical, nei quali \`e possibile richiedere ore di lavoro maggiori delle 8 
standard.
\end{itemize}

Per i precedenti punti avremo che per rappresentare un giorno saranno
necessari\\ $day_{width} = \textbf{7.2cm}$. In totale saranno rappresentabili 
almeno $days$ giorni correttamente:
\begin{displaymath}
	days = \left \lfloor \frac{\frac{5}{6}PrintableArea_{width}
	}{day_{width}}\right \rfloor
\end{displaymath}


\item[giorno] \quad
\begin{itemize}
  \item fissiamo la dimensione del gap fra una linea verticale tratteggiata
e la successiva uguale a \textbf{5mm}.
  \item Si rappresentano \textbf{7} giorni per settimana
\end{itemize}

Per i precedenti punti avremo che per rappresentare una settimana saranno
necessari\\ $week_{width} = \textbf{3.5cm}$. In totale saranno rappresentabili 
almeno $weeks$ settimane correttamente:
\begin{displaymath}
	weeks = \left \lfloor \frac{\frac{5}{6}PrintableArea_{width}
	}{week_{width}}\right \rfloor
\end{displaymath}


\item[settimana] \quad
\begin{itemize}
  \item fissiamo la dimensione del gap fra una linea verticale tratteggiata
e la successiva uguale a \textbf{1cm}.
  \item Si rappresentano \textbf{4} settimane per mese
\end{itemize}

Per i precedenti punti avremo che per rappresentare un mese saranno
necessari\\ $month_{width} = \textbf{4.0cm}$. In totale saranno rappresentabili 
almeno $months$ mesi correttamente:
\begin{displaymath}
	months = \left \lfloor \frac{\frac{5}{6}PrintableArea_{width}
	}{month_{width}}\right \rfloor
\end{displaymath}


\item[mese] \quad
\begin{itemize}
  \item fissiamo la dimensione del gap fra una linea verticale tratteggiata
e la successiva uguale a \textbf{1cm}.
  \item Si rappresentano \textbf{12} settimane per mese
\end{itemize}

Per i precedenti punti avremo che per rappresentare un mese saranno
necessari\\ $year_{width} = \textbf{12.0cm}$. In totale saranno rappresentabili 
almeno $years$ anni correttamente:
\begin{displaymath}
	years = \left \lfloor \frac{ \frac{5}{6}PrintableArea_{width}
	}{year_{width}}\right \rfloor
\end{displaymath}


\item[anno] \quad
\begin{itemize}
  \item fissiamo la dimensione del gap fra una linea verticale tratteggiata
e la successiva uguale a $year_{width}\textbf{3cm}$.
\end{itemize}

In totale saranno rappresentabili almeno $years$ anni correttamente:
\begin{displaymath}
	years = \left \lfloor \frac{\frac{5}{6}PrintableArea_{width}
	}{year_{width}}\right \rfloor
\end{displaymath}

\end{description}

\subsection{Metriche sullo spazio occupato dal diagramma WBS}
Il diagramma WBS ha a disposizione lo spazio offerto da una pagina per la
stampa su carta (margini di stampa esclusi), si vuole quindi dare dei limiti dei parametri del diagramma entro i quali si garantisce che questo non uscir\'a dall'area di stampa. I parametri che andremo ad utilizzare sono i seguenti:
\begin{itemize}
	\item \emph{foglie} $leaves$: l'insieme delle foglie. Dato che il diagramma WBS \'e una struttura dati ad albero ad ariet\'a non fissata, le foglie sono i task che non presentano sotto task figli,
	\item \emph{livelli} $levels$: l'insieme dei livelli. Il numero di livelli $|levels|$ viene definito come il massimo numero di nodi che si incontrano percorrendo le relazioni nel solo verso padre-figlio, partendo dalla radice del diagramma,
	\item \emph{altezza e larghezza nodi} $node_{height}$, $node_{width}$: le due dimensioni dei box contententi i task dipendono dalla quantità di informazioni che si vuole inserire,
	\item \emph{margine orizzontale tra nodi} $HorizontalMargin$: la distanza minima che si deve avere tra due nodi adiacenti, fissata a \textbf{3mm} ma modificabile dalle opzioni di configurazione,
	\item \emph{margine verticale tra nodi} $VerticalMargin$: la distanza tra i livelli
	dell'albero, nei quali devono rientrare i rami di relazione tra i task, di default \'e fissata a \textbf{6mm} ma modificabile dalle opzioni di configurazione,
	\item \emph{area stampabile} $PrintableArea_{width}$,
	$PrintableArea_{height}$: rappresentano le dimensioni dell'area stampabile del foglio, rispettivamente larghezza e altezza, quindi considerando gi\'a eliminati i margini di stampa.
\end{itemize}

Stabiliti i parametri passiamo a dichiarare le disequazioni rappresentanti i limiti entro i quali diamo garanzie:
\begin{itemize}
	\item garantiamo che un diagramma WBS rientri, per \emph{larghezza}, nell'area di stampa se:
	$$ |leaves| \times node_{width} + (|leaves| - 1) \times HorizontalMargin \leq PrintableArea_{width} $$
	\item garantiamo che un diagramma WBS rientri, per \emph{altezza}, nell'area di stampa se:
	$$ |levels| \times node_{height} + (|levels| - 1) \times VerticalMargin \leq PrintableArea_{height} $$
\end{itemize}
Intuitivamente questi limiti tutelano i casi peggiori, per esempio quando si
presenta un albero completo, il cui spazio occupato \`e il limite massimo espresso dalle formule.

\subsection{Metriche sullo spazio occupato dal diagramma Task Network}
Il diagramma task network pu\`o essere associato alla struttura dati grafo
aciclico (esprimiamo quindi le metriche solo per i casi \emph{well formed}). Come anche per il diagramma WBS, si vuole dare dei limiti sui parametri della task network entro i quali assicuriamo la stampa su carta nei margini fissati. I parametri che andremo ad utilizzare sono i seguenti:
\begin{itemize}
	\item \emph{larghezza dei nodi di inizio e fine} $start_{width}$, $end_{width}$: misurano la larghezza occupata dai nodi di inizio e fine,
	\item \emph{livelli} $levels$: l'insieme dei livelli. Ad ogni nodo viene associato un numero di livello e i nodi che hanno lo stesso numero si dicono appartenere allo stesso livello. L'etichettamento dei nodi col numero di livello inizia associando al nodo di inizio il valore zero e successivamente ad ogni nodo viene associato il numero di livello massimo tra quelli dei suoi padri incrementato di uno. Il numero di livelli $|levels|$ pu\`o quindi essere calcolato, dopo la fase di etichettamento, come il livello del nodo finale decrementato di uno,
	\item \emph{nodi di un livello fissato i} $level_i$:
	l'insieme di tutti i nodi appartenenti al livello $i$. La cardinalit\`a di questo insieme $|level_i|$ viene utilizzata per valutare che dimensioni pu\`o raggiungere in altezza il livello, e quindi il grafo,
	\item \emph{altezza e larghezza nodi} $node_{height}$, $node_{width}$: 
	analogo ai parametri dei diagrammi WBS,
	\item \emph{margine orizzontale tra nodi} $HorizontalMargin$ ($HM$): 
	analogo ai parametri dei diagrammi WBS,
	\item \emph{margine verticale tra nodi} $VerticalMargin$ ($VM$): 
	la distanza verticale minima tra i nodi, di default \`e fissata a \textbf{3mm} ma modificabile dalle opzioni di configurazione,
	\item \emph{archi che cambiano quota} $ChangingHeightArcs$ ($CHA$): 
	l'insieme di tutti gli archi che cambiano di quota, in quanto non riescono a raggiungere direttamente il nodo di arrivo. Il numero degli archi $|ChangingHeightArcs|$ che hanno questo comportamento \`e importante perch\'e ogni cambio di traiettoria viene fatto nello spazio tra due livelli adiacenti e non deve coincidere tra più archi al fine di evitare rappresentazioni ambigue. Il margine orizzontale tra due livelli aumenta quindi all'aumentare della cardinalit\`a di questo insieme,
	\item \emph{archi che attraversano un livello fissato i} $ArcsTroughtLevel_i$ ($ATL_i$):
	l'insieme di tutti gli archi che attraversano il livello $i$ per arrivare a un livello successivo o per entrare in una sotto attivit\`a di un task del livello $i$, come per i $ChangingHeightArcs$ il numero di questi archi $|ArcsTroughtLevel_i|$ \`e importante in quanto vanno ad aumentare l'altezza del livello che attraversano, e quindi la potenzale altezza dell'intero grafo,
	\item \emph{margini tra archi} $ArcMargin$ ($AM$) : indica la distanza minima tra due archi o tra un arco e un nodo, per default fissata a \textbf{2mm} ma modificabile dalel opzioni di configurazione,
	\item \emph{area stampabile} $PrintableArea_{width}$ ($PA_{width}$), $PrintableArea_{height}$ ($PA_{height}$): rappresentano le dimensioni dell'area stampabile del foglio, rispettivamente larghezza e altezza, quindi considerando gi\`a eliminati i margini di stampa.
\end{itemize}

Stabiliti i parametri passiamo a dichiarare le disequazioni rappresentanti i limiti entro i quali diamo garanzie:
\begin{itemize}
	\item garantiamo che un diagramma task network rientri, per \emph{larghezza}, nell'area di stampa se:
	$$ start_{width} + end_{width} + |levels| \times node_{width} + 	
	(|levels|+1) \times HM + |CHA| \times AM
	\leq PA_{width} $$
	\item garantiamo che un diagramma task network rientri, per \emph{altezza}, nell'area di stampa se:
	$$ max_{1 \leq i \leq |levels|}\{ |level_i| \times node_{height} +
	(|level_i|-1) \times VM + |ATL_i| \times AM \} 
	\leq PA_{height} $$
\end{itemize}


