\chapter*{Release \textbf{1.2}}

\chapter{Requisiti obbligatori}

\section{Generali (2.1)}
Come requisiti fondamentali, \textbf{PMango 3.0} sar\`a visualizzabile e
usabile con le ultime versioni \emph{Internet Explorer 8} e \emph{Mozilla
Firefox 3.0}.

Le nostre modifiche e aggiunte saranno distribuite senza costi di licenza, in
quanto si tratta di estensioni di un progetto GPL

Ci assumiamo la responsabilit\`a di essere conformi ai punti \emph{d), e)}.

\section{Diagrammi WBS, Gantt e Task Network (2.2)}
\begin{itemize}
  \item[a)] implementato nello use case \textbf{\ref{seq:showProjectPage}}.
  \item[b)] implementato negli use case \textbf{\ref{seq:makeNodeTaskbox}} e in
  \textbf{\ref{seq:makeGanttTaskbox}}.
  \item[c)] implementato nello use case \textbf{\ref{seq:showUserOptions}}
\end{itemize}

\section{Diagrammi specifici (2.3)}
\begin{itemize}
  \item[a)] implementato nello use case \textbf{\ref{seq:showUserOptions}}
  \footnote{replace every ShowUserOption reference with the relative
  generalization}
  \item[b)] implementato nello use case
  \textbf{\ref{seq:createCriticalPathTable}}
  \item[c)] implementato nello use case \textbf{\ref{seq:showUserOptions}}
  \item[d)] implementato nello use case \textbf{\ref{seq:showUserOptions}} e
  descritto in modo dettagliato nella sezione
  \textbf{\ref{subsec:UserOptionInstances}} del documento
  \textbf{Domain Model}
  \item[e)] realizzato nello use case \textbf{\ref{seq:showUserOptions}}
\end{itemize}

\section{Generazione di immagini e doc (2.4)}
\begin{itemize}
  \item[a)] realizzato negli use case \textbf{\ref{seq:refreshChart},
  \ref{seq:commonsMakePDF}}
  \item[b)] realizzato nello use case \textbf{\ref{seq:addToReportUserAction}}
  \item[c)] realizzato nello use case \textbf{\ref{seq:showUserOptions}} e
  descritto in modo dettagliato nella sezione
  \textbf{\ref{subsec:UserOptionInstances}}  del documento \textbf{Domain Model}
  \item[d)] vedi punto \emph{a)}
  \item[e)] realizzato nello use case \textbf{\ref{seq:openInNewWindow}}
\end{itemize}

\section{Documentazione (2.5)}
Ci assumiamo la responsabilit\`a di essere coerenti a quanto richiesto nei
punti \emph{a), b), c)} nei momenti in cui verranno effettivamente implementati.

\chapter{Semplificazioni, Metriche}
\section{Semplificazioni e requisiti aggiuntivi}
\begin{itemize}
  \item[a)] il nostro gruppo \textbf{non} prevede lo sviluppo di requisiti
  aggiuntivi, preferendo implementare correttamente il processo di sviluppo 
  adottato per raggiungere i requisiti richiesti dal committente.
  \item[b)] abbiamo deciso di creare \textbf{solo} oggetti \textbf{gif} per
  usarli in modo interscambiabile sia nella visualizzazione da browser web, sia
  per aggiungerli in documenti PDF. Questo ci porta alcuni vantaggi:
  \begin{itemize}
    \item ci interfacciamo con una sola libreria, avendo cosi modo di capirne a
    fondo il comportamento e eventualmente aggiungere quelle funzionalit\`a di
    helper che potrebbero servirci, ma che attualmente non vengono fornite.
    \item ci riduce il carico di lavoro, questo non preclude che se arriviamo
    in anticipo con un prodotto finito e che rispetta la specifica richiesta,
    potremo proporre una integrazione dell'offerta sviluppando le
    funzionalit\`a native per la rappresentazione in PDF.
    \end{itemize}
\end{itemize}

\section{Metriche}

\subsection{Metrica sulla grana temporale di un Gantt}
Vogliamo fissare un limite minimo di informazioni che vengono rappresentate
nella stampa su \emph{carta} di un \emph{Gantt chart} in base alla grana
temporale scelta.

Facciamo queste assunzioni:
\begin{itemize}
  \item la stampa del \emph{Gantt chart} \`e \textbf{landscape}.
  \item \emph{PrintableArea} \`e l'area del foglio A4 su cui \`e effettivamente
  possibile stampare. In \emph{PrintableArea} non rientrano i quattro margini
  di stampa della stampante che viene utilizzata per la stampa.
  
  Su \emph{PrintableArea} sono definite tutte le classiche operazioni di misura
  di cui potremo avere bisogno in seguito (esempio: misurare un lato di
  \emph{PrintableArea} \`e un operazione possibile)
  
  \item Per la larghezza della colonna di sinistra del diagramma che ospita gli
  id e i nomi dei vari \emph{Task} si suppone valga questa uguaglianza:
  \begin{displaymath}
  	width_{left\_column} = \frac{LandscapeMeasure(PrintableArea)}{6}
  \end{displaymath}
  
   \item Si suppone che l'utente non abbia selezionato le \emph{UserOption} per
   avere informazioni sulle risorse sulla destra della \emph{GanttTaskbox}.

	\item Sia $GrainAvailable = \lbrace ora
	, giorno, settimana, mese, anno	\rbrace$. Definiamo la relazione $\sqsubset$
	come:
	\begin{displaymath}
	\sqsubset = \lbrace (a, b) \in GrainAvailable \times GrainAvailable : b \quad
	aggregates \quad a
	\rbrace
	\end{displaymath}
	l'operatore $b \quad aggregates \quad a$ esprime che $b$ \`e composto da alcuni
	$a$.
	
	Usando la precedente relazione vale:
	\begin{displaymath}
  	 ora \sqsubset giorno \sqsubset settimana \sqsubset mese \sqsubset anno
  	\end{displaymath}
\end{itemize}

Definiamo la metrica in base alle possibili grane temporali:
\begin{description}
\item[ora] \quad
\begin{itemize}
  \item fissiamo la dimensione del gap fra una linea verticale tratteggiata
e la successiva uguale a \textbf{3mm}.
  \item Si rappresentano \textbf{24} ore per giorno, per supportare progetti
  mission critical, nei quali \`e possibile richiedere ore di lavoro maggiori delle 8 
standard.
\end{itemize}

Per i precedenti punti avremo che per rappresentare un giorno saranno
necessari\\ $day_{width} = \textbf{7.2cm}$. In totale saranno rappresentabili 
almeno $days$ giorni correttamente:
\begin{displaymath}
	days = \left \lfloor \frac{measure \left (\frac{5}{6}PrintableArea
	\right)}{day_{width}}\right \rfloor
\end{displaymath}
con $measure$ l'operatore che restituisce la misura in cm dell'argomento.

\item[giorno] \quad
\begin{itemize}
  \item fissiamo la dimensione del gap fra una linea verticale tratteggiata
e la successiva uguale a \textbf{5mm}.
  \item Si rappresentano \textbf{7} giorni per settimana
\end{itemize}

Per i precedenti punti avremo che per rappresentare una settimana saranno
necessari\\ $week_{width} = \textbf{3.5cm}$. In totale saranno rappresentabili 
almeno $weeks$ settimane correttamente:
\begin{displaymath}
	weeks = \left \lfloor \frac{measure \left (\frac{5}{6}PrintableArea
	\right)}{week_{width}}\right \rfloor
\end{displaymath}
con $measure$ l'operatore che restituisce la misura in cm dell'argomento.

\item[settimana] \quad
\begin{itemize}
  \item fissiamo la dimensione del gap fra una linea verticale tratteggiata
e la successiva uguale a \textbf{1cm}.
  \item Si rappresentano \textbf{4} settimane per mese
\end{itemize}

Per i precedenti punti avremo che per rappresentare un mese saranno
necessari\\ $month_{width} = \textbf{4.0cm}$. In totale saranno rappresentabili 
almeno $months$ mesi correttamente:
\begin{displaymath}
	months = \left \lfloor \frac{measure \left (\frac{5}{6}PrintableArea
	\right)}{month_{width}}\right \rfloor
\end{displaymath}
con $measure$ l'operatore che restituisce la misura in cm dell'argomento.

\item[mese] \quad
\begin{itemize}
  \item fissiamo la dimensione del gap fra una linea verticale tratteggiata
e la successiva uguale a \textbf{1cm}.
  \item Si rappresentano \textbf{12} settimane per mese
\end{itemize}

Per i precedenti punti avremo che per rappresentare un mese saranno
necessari\\ $year_{width} = \textbf{12.0cm}$. In totale saranno rappresentabili 
almeno $years$ anni correttamente:
\begin{displaymath}
	years = \left \lfloor \frac{measure \left (\frac{5}{6}PrintableArea
	\right)}{year_{width}}\right \rfloor
\end{displaymath}
con $measure$ l'operatore che restituisce la misura in cm dell'argomento.

\item[anno] \quad
\begin{itemize}
  \item fissiamo la dimensione del gap fra una linea verticale tratteggiata
e la successiva uguale a $year_{width}\textbf{3cm}$.
\end{itemize}

In totale saranno rappresentabili almeno $years$ anni correttamente:
\begin{displaymath}
	years = \left \lfloor \frac{measure \left (\frac{5}{6}PrintableArea
	\right)}{year_{width}}\right \rfloor
\end{displaymath}
con $measure$ l'operatore che restituisce la misura in cm dell'argomento.
\end{description}






