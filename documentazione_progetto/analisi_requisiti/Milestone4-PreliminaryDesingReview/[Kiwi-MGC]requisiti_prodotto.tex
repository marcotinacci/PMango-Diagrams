\chapter*{Release \textbf{1.1}}

\chapter{Requisiti obbligatori}

\section{Generali (2.1)}
Come requisiti fondamentali, \textbf{PMango 3.0} sar\`a visualizzabile e
usabile con le ultime versioni \emph{Internet Explorer 8} e \emph{Mozilla
Firefox 3.0}.

Le nostre modifiche e aggiunte saranno distribuite senza costi di licenza, in
quanto si tratta di estensioni di un progetto GPL

Ci assumiamo la responsabilit\`a di essere conformi ai punti \emph{d), e)}.

\section{Diagrammi WBS, Gantt e Task Network (2.2)}
\begin{itemize}
  \item[a)] implementato nello use case \textbf{\ref{seq:showProjectPage}}.
  \item[b)] implementato negli use case \textbf{\ref{seq:makeNodeTaskbox}} e in
  \textbf{\ref{seq:makeGanttTaskbox}}.
  \item[c)] implementato nello use case \textbf{\ref{seq:showUserOptions}}
\end{itemize}

\section{Diagrammi specifici (2.3)}
\begin{itemize}
  \item[a)] implementato nello use case \textbf{\ref{seq:showUserOptions}}
  \footnote{replace every ShowUserOption reference with the relative
  generalization}
  \item[b)] implementato nello use case
  \textbf{\ref{seq:createCriticalPathTable}}
  \item[c)] implementato nello use case \textbf{\ref{seq:showUserOptions}}
  \item[d)] implementato nello use case \textbf{\ref{seq:showUserOptions}} e
  descritto in modo dettagliato nella sezione
  \textbf{\ref{subsec:UserOptionInstances}} del documento
  \textbf{Domain Model}
  \item[e)] realizzato nello use case \textbf{\ref{seq:showUserOptions}}
\end{itemize}

\section{Generazione di immagini e doc (2.4)}
\begin{itemize}
  \item[a)] realizzato negli use case \textbf{\ref{seq:refreshChart},
  \ref{seq:commonsMakePDF}}
  \item[b)] realizzato nello use case \textbf{\ref{seq:addToReportUserAction}}
  \item[c)] realizzato nello use case \textbf{\ref{seq:showUserOptions}} e
  descritto in modo dettagliato nella sezione
  \textbf{\ref{subsec:UserOptionInstances}}  del documento \textbf{Domain Model}
  \item[d)] vedi punto \emph{a)}
  \item[e)] realizzato nello use case \textbf{\ref{seq:openInNewWindow}}
\end{itemize}

\section{Documentazione (2.5)}
Ci assumiamo la responsabilit\`a di essere coerenti a quanto richiesto nei
punti \emph{a), b), c)} nei momenti in cui verranno effettivamente implementati.

\chapter{Semplificazioni, Metriche}
\section{Semplificazioni e requisiti aggiuntivi}
\begin{itemize}
  \item[a)] il nostro gruppo \textbf{non} prevede lo sviluppo di requisiti
  aggiuntivi, preferendo implementare correttamente il processo di sviluppo adottato per
  raggiungere i requisiti richiesti dal committente.
  \item[b)] abbiamo deciso di creare \textbf{solo} oggetti \textbf{gif} per
  usarli in modo interscambiabile sia nella visualizzazione da browser web, sia
  per aggiungerli in documenti PDF. Questo ci porta alcuni vantaggi:
  \begin{itemize}
    \item ci interfacciamo con una sola libreria, avendo cosi modo di capirne a
    fondo il comportamento e eventualmente aggiungere quelle funzionalit\`a di
    helper che potrebbero servirci, ma che attualmente non vengono fornite.
    \item ci riduce il carico di lavoro, questo non preclude che se arriviamo
    in anticipo con un prodotto finito e che rispetta la specifica richiesta,
    potremo proporre una integrazione dell'offerta sviluppando le
    funzionalit\`a native per la rappresentazione in PDF.
    \end{itemize}
\end{itemize}

\section{Metriche}