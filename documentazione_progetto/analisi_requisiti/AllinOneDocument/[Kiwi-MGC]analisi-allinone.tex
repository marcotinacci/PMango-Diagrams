\documentclass[a4paper, 12pt]{report}

\usepackage{charter}
\usepackage{makeidx}
\usepackage{fancyhdr}
\usepackage{hyperref}
\usepackage[utf8]{inputenc}
\usepackage{graphicx}
\usepackage[left=2cm, right=2cm]{geometry}
\usepackage{latexsym}
\usepackage{amsmath, amsthm, amssymb}
\usepackage{rotating}


\begin{titlepage}
\title{Documento di Analisi}
\author{Release 1.2}
\date{\today \\Firenze \\\begin{figure}[h] \centering
\includegraphics[width=0.2\textwidth]{../../images/logokiwi.png} \end{figure} }
\end{titlepage}

\pagestyle{fancy}

\begin{document}

\maketitle

\section*{Approvazione, redazione, lista distribuzione}
\begin{table}[h!]
  \begin{center}
    \begin{tabular}{| l | l | p{60mm} |}
    \hline
    \textbf{approvato da} & \textbf{il giorno} & \textbf{firma} \\
	\hline    
	Marco Tinacci & \today &  \\
    \hline
    \end{tabular}
  \end{center}
\end{table}

\begin{table}[h!]
  \begin{center}
    \begin{tabular}{| l | l | p{60mm} |}
    \hline
    \textbf{redatto da} & \textbf{il giorno} & \textbf{firma} \\
    \hline
    Francesco Calabri & \today &  \\
    \hline
	Manuele Paulantonio & \today &  \\
    \hline    
	Massimo Nocentini & \today &  \\
    \hline
    \end{tabular}
  \end{center}
\end{table}

\begin{table}[h!]
  \begin{center}
    \begin{tabular}{| l | l | p{60mm} |}
    \hline
    \textbf{distribuito a} & \textbf{il giorno} & \textbf{firma} \\
	\hline    
	Daniele Poggi & \today &  \\
    \hline
	Niccol\'o Rogai & \today &  \\
    \hline
	Marco Tinacci & \today &  \\
    \hline
    \end{tabular}
  \end{center}
\end{table}

\tableofcontents

\newpage

\section*{Introduzione}
Questo documento riassume il processo di analisi con tutti i documenti che ogni
singola componente del processo ha prodotto.

Abbiamo cercato di approcciare il problema dell'analisi dei requisiti seguendo
le idee del modello agile \textbf{ICONIX}. Questi sono i passi che abbiamo
sviluppato in ordine cronologico:
\begin{enumerate}
  \item brainstroming iniziale sul problema leggendo il documento di specifica,
  rilasciato dal committente. Questa fase ha portato alla scoperta dei concetti
  che il problema richiede di trattare e li abbiamo catturati nel documento
  \emph{Domain Model}. Questo documento \`e per definizione non corretto,
  infatti con gli step successivi siamo stati in grado di raffinare la sua
  granularit\`a, aggiungendo concetti che non erano stati modellati e
  cancellando concetti simili, fattorizzandoli.
  \item stesura degli \emph{use cases}, cercando di catturare i comportamenti e
  le sequenze di interazione che un client vuole eseguire sul problema. Questo ci
  ha portato ad una prima definizione un po troppo formale, dedicata pi\`u al 
  ``come'' il sistema esegue il comportamento, invece che foaclizzarsi sul 
  ``cosa'' avviene nei comportamenti. Quindi abbiamo corretto il tiro,
  formulando degli use cases che modellano l'insieme delle posssibili
  interazioni richieste dal committente, stando molto attenti a definire sia il
  \emph{basic course}, ovvero la sequenza di azioni nel caso che non si creino
  problemi durante la sequenza, sia un \emph{alternate course}, ovvero
  modellare cosa accade se si verifica un errore. 
  
  Inoltre abbiamo cercato di raggruppare use case in pacchetti relativi al
  contesto delle azioni che venivano modellate.
\end{enumerate}

Questo documento \`e diviso in tre parti indipendenti:
\begin{description}
\item[requisiti del prodotto] possiamo vedere questo documento come una mappa:
mette in relazione le richieste espresse nei punti del capitolato di appalto
con i nostri oggetti di analisi, quindi concetti del \emph{Domain Model} e
\emph{use cases}. 

Infatti abbiamo cercato di dargli proprio questo taglio: abbiamo due capitoli
riferiti rispettivamente ai requisiti obbligatori e alle semplificazioni,
metriche. Per ogni requisito abbiamo riportato la dicitura del documento di
appalto, scrivendo poi per ogni richiesta, il relativo concetto della nostra
analisi che la soddisfa.
\item[Domain Model] questo \`e il dizionario di concetti che le successive fasi
di sviluppo del problema condividono. In questo documento vengono catturari
tutti i concetti e le relazioni di \emph{aggregazione} e
\emph{generalizzazione} che esistono fra di essi. Sar\`a la base del diagramma
delle classi, che verr\`a sviluppato nella fase di progettazione.

La sua struttura \`e una sequenza di descrizioni di piccoli spaccati presi dal
diagramma principale, per evidenziarne i punti fondamentali e mettendoli
in relazione con altre parti del diagramma, senza doverle discrivere tutte
insieme.
\item[Use cases] questo documento invece cattura i comportamenti attesi che si
possono eseguire a runtime con il problema. Abbiamo diviso questo documento in
quattro parti, ognuna in relazione con la divisione in pacchetti degli stessi
use case. Abbiamo una \emph{commons} e le successive tre, una per ogni
\emph{Chart} che dovremo sviluppare. Per ogni parte riportiamo un diagramma che
esprime con notazione UML le relazioni di \emph{precedes}, e successivamente i
paragrafi che costruiscono veramente lo use case.
\end{description}

\subsection*{Alcune definizioni}

Descrizione dell'acronimo: \textbf{cdns} sta per ''\textbf{c}ome
\textbf{d}escritto \textbf{n}ella \textbf{s}pecifica''.

\part{Requisiti del prodotto}

\chapter*{Release \textbf{1.2}}

\chapter{Requisiti obbligatori}

\section{Generali (2.1)}
Come requisiti fondamentali, \textbf{PMango 3.0} sar\`a visualizzabile e
usabile con le ultime versioni \emph{Internet Explorer 8} e \emph{Mozilla
Firefox 3.0}.

Le nostre modifiche e aggiunte saranno distribuite senza costi di licenza, in
quanto si tratta di estensioni di un progetto GPL

Ci assumiamo la responsabilit\`a di essere conformi ai punti \emph{d), e)}.

\section{Diagrammi WBS, Gantt e Task Network (2.2)}
\begin{itemize}
  \item[a)] implementato nello use case \textbf{\ref{seq:showProjectPage}}.
  \item[b)] implementato negli use case \textbf{\ref{seq:makeNodeTaskbox}} e in
  \textbf{\ref{seq:makeGanttTaskbox}}.
  \item[c)] implementato nello use case \textbf{\ref{seq:showUserOptions}}
\end{itemize}

\section{Diagrammi specifici (2.3)}
\begin{itemize}
  \item[a)] implementato nello use case \textbf{\ref{seq:showUserOptions}}
  \footnote{replace every ShowUserOption reference with the relative
  generalization}
  \item[b)] implementato nello use case
  \textbf{\ref{seq:createCriticalPathTable}}
  \item[c)] implementato nello use case \textbf{\ref{seq:showUserOptions}}
  \item[d)] implementato nello use case \textbf{\ref{seq:showUserOptions}} e
  descritto in modo dettagliato nella sezione
  \textbf{\ref{subsec:UserOptionInstances}} del documento
  \textbf{Domain Model}
  \item[e)] realizzato nello use case \textbf{\ref{seq:showUserOptions}}
\end{itemize}

\section{Generazione di immagini e doc (2.4)}
\begin{itemize}
  \item[a)] realizzato negli use case \textbf{\ref{seq:refreshChart},
  \ref{seq:commonsMakePDF}}
  \item[b)] realizzato nello use case \textbf{\ref{seq:addToReportUserAction}}
  \item[c)] realizzato nello use case \textbf{\ref{seq:showUserOptions}} e
  descritto in modo dettagliato nella sezione
  \textbf{\ref{subsec:UserOptionInstances}}  del documento \textbf{Domain Model}
  \item[d)] vedi punto \emph{a)}
  \item[e)] realizzato nello use case \textbf{\ref{seq:openInNewWindow}}
\end{itemize}

\section{Documentazione (2.5)}
Ci assumiamo la responsabilit\`a di essere coerenti a quanto richiesto nei
punti \emph{a), b), c)} nei momenti in cui verranno effettivamente implementati.

\chapter{Semplificazioni, Metriche}
\section{Semplificazioni e requisiti aggiuntivi}
\begin{itemize}
  \item[a)] il nostro gruppo \textbf{non} prevede lo sviluppo di requisiti
  aggiuntivi, preferendo implementare correttamente il processo di sviluppo 
  adottato per raggiungere i requisiti richiesti dal committente.
  \item[b)] abbiamo deciso di creare \textbf{solo} oggetti \textbf{gif} per
  usarli in modo interscambiabile sia nella visualizzazione da browser web, sia
  per aggiungerli in documenti PDF. Questo ci porta alcuni vantaggi:
  \begin{itemize}
    \item ci interfacciamo con una sola libreria, avendo cosi modo di capirne a
    fondo il comportamento e eventualmente aggiungere quelle funzionalit\`a di
    helper che potrebbero servirci, ma che attualmente non vengono fornite.
    \item ci riduce il carico di lavoro, questo non preclude che se arriviamo
    in anticipo con un prodotto finito e che rispetta la specifica richiesta,
    potremo proporre una integrazione dell'offerta sviluppando le
    funzionalit\`a native per la rappresentazione in PDF.
    \end{itemize}
\end{itemize}

\section{Metriche}

\subsection{Metrica sulla grana temporale di un Gantt}
Vogliamo fissare un limite minimo di informazioni che vengono rappresentate
nella stampa su \emph{carta} di un \emph{Gantt chart} in base alla grana
temporale scelta.

Facciamo queste assunzioni:
\begin{itemize}
  \item la stampa del \emph{Gantt chart} \`e \textbf{landscape}.
  \item \emph{PrintableArea} \`e l'area del foglio A4 su cui \`e effettivamente
  possibile stampare. In \emph{PrintableArea} non rientrano i quattro margini
  di stampa della stampante che viene utilizzata per la stampa.
  
  \item area stampabile $PrintableArea_{width}$ , $PrintableArea_{height}$:
  rappresentano le dimensioni dell'area stampabile del foglio, rispettivamente 
  larghezza e altezza, quindi considerando gia eliminati i margini di stampa.
  
  \item Per la larghezza della colonna di sinistra del diagramma che ospita gli
  id e i nomi dei vari \emph{Task} si suppone valga questa uguaglianza:
  \begin{displaymath}
  	left\_column_{width} = \frac{PrintableArea_{width}}{6}
  \end{displaymath}
  
   \item Si suppone che l'utente non abbia selezionato le \emph{UserOption} per
   avere informazioni sulle risorse sulla destra della \emph{GanttTaskbox}.

	\item Sia $GrainAvailable = \lbrace ora
	, giorno, settimana, mese, anno	\rbrace$. Definiamo la relazione $\sqsubset$
	come:
	\begin{displaymath}
	\sqsubset = \lbrace (a, b) \in GrainAvailable \times GrainAvailable : b \quad
	aggregates \quad a
	\rbrace
	\end{displaymath}
	l'operatore $b \quad aggregates \quad a$ esprime che $b$ \`e composto da alcuni
	$a$.
	
	Usando la precedente relazione vale:
	\begin{displaymath}
  	 ora \sqsubset giorno \sqsubset settimana \sqsubset mese \sqsubset anno
  	\end{displaymath}
\end{itemize}

Definiamo la metrica in base alle possibili grane temporali:
\begin{description}
\item[ora] \quad
\begin{itemize}
  \item fissiamo la dimensione del gap fra una linea verticale tratteggiata
e la successiva uguale a \textbf{3mm}.
  \item Si rappresentano \textbf{24} ore per giorno, per supportare progetti
  mission critical, nei quali \`e possibile richiedere ore di lavoro maggiori delle 8 
standard.
\end{itemize}

Per i precedenti punti avremo che per rappresentare un giorno saranno
necessari\\ $day_{width} = \textbf{7.2cm}$. In totale saranno rappresentabili 
almeno $days$ giorni correttamente:
\begin{displaymath}
	days = \left \lfloor \frac{\frac{5}{6}PrintableArea_{width}
	}{day_{width}}\right \rfloor
\end{displaymath}


\item[giorno] \quad
\begin{itemize}
  \item fissiamo la dimensione del gap fra una linea verticale tratteggiata
e la successiva uguale a \textbf{5mm}.
  \item Si rappresentano \textbf{7} giorni per settimana
\end{itemize}

Per i precedenti punti avremo che per rappresentare una settimana saranno
necessari\\ $week_{width} = \textbf{3.5cm}$. In totale saranno rappresentabili 
almeno $weeks$ settimane correttamente:
\begin{displaymath}
	weeks = \left \lfloor \frac{\frac{5}{6}PrintableArea_{width}
	}{week_{width}}\right \rfloor
\end{displaymath}


\item[settimana] \quad
\begin{itemize}
  \item fissiamo la dimensione del gap fra una linea verticale tratteggiata
e la successiva uguale a \textbf{1cm}.
  \item Si rappresentano \textbf{4} settimane per mese
\end{itemize}

Per i precedenti punti avremo che per rappresentare un mese saranno
necessari\\ $month_{width} = \textbf{4.0cm}$. In totale saranno rappresentabili 
almeno $months$ mesi correttamente:
\begin{displaymath}
	months = \left \lfloor \frac{\frac{5}{6}PrintableArea_{width}
	}{month_{width}}\right \rfloor
\end{displaymath}


\item[mese] \quad
\begin{itemize}
  \item fissiamo la dimensione del gap fra una linea verticale tratteggiata
e la successiva uguale a \textbf{1cm}.
  \item Si rappresentano \textbf{12} settimane per mese
\end{itemize}

Per i precedenti punti avremo che per rappresentare un mese saranno
necessari\\ $year_{width} = \textbf{12.0cm}$. In totale saranno rappresentabili 
almeno $years$ anni correttamente:
\begin{displaymath}
	years = \left \lfloor \frac{ \frac{5}{6}PrintableArea_{width}
	}{year_{width}}\right \rfloor
\end{displaymath}


\item[anno] \quad
\begin{itemize}
  \item fissiamo la dimensione del gap fra una linea verticale tratteggiata
e la successiva uguale a $year_{width}\textbf{3cm}$.
\end{itemize}

In totale saranno rappresentabili almeno $years$ anni correttamente:
\begin{displaymath}
	years = \left \lfloor \frac{\frac{5}{6}PrintableArea_{width}
	}{year_{width}}\right \rfloor
\end{displaymath}

\end{description}

\subsection{Metriche sullo spazio occupato dal diagramma WBS}
Il diagramma WBS ha a disposizione lo spazio offerto da una pagina per la
stampa su carta (margini di stampa esclusi), si vuole quindi dare dei limiti dei parametri del diagramma entro i quali si garantisce che questo non uscir\'a dall'area di stampa. I parametri che andremo ad utilizzare sono i seguenti:
\begin{itemize}
	\item \emph{foglie} $leaves$: l'insieme delle foglie. Dato che il diagramma WBS \'e una struttura dati ad albero ad ariet\'a non fissata, le foglie sono i task che non presentano sotto task figli,
	\item \emph{livelli} $levels$: l'insieme dei livelli. Il numero di livelli $|levels|$ viene definito come il massimo numero di nodi che si incontrano percorrendo le relazioni nel solo verso padre-figlio, partendo dalla radice del diagramma,
	\item \emph{altezza e larghezza nodi} $node_{height}$, $node_{width}$: le due dimensioni dei box contententi i task dipendono dalla quantità di informazioni che si vuole inserire,
	\item \emph{margine orizzontale tra nodi} $HorizontalMargin$: la distanza minima che si deve avere tra due nodi adiacenti, fissata a \textbf{3mm} ma modificabile dalle opzioni di configurazione,
	\item \emph{margine verticale tra nodi} $VerticalMargin$: la distanza tra i livelli
	dell'albero, nei quali devono rientrare i rami di relazione tra i task, di default \'e fissata a \textbf{6mm} ma modificabile dalle opzioni di configurazione,
	\item \emph{area stampabile} $PrintableArea_{width}$,
	$PrintableArea_{height}$: rappresentano le dimensioni dell'area stampabile del foglio, rispettivamente larghezza e altezza, quindi considerando gi\'a eliminati i margini di stampa.
\end{itemize}

Stabiliti i parametri passiamo a dichiarare le disequazioni rappresentanti i limiti entro i quali diamo garanzie:
\begin{itemize}
	\item garantiamo che un diagramma WBS rientri, per \emph{larghezza}, nell'area di stampa se:
	$$ |leaves| \times node_{width} + (|leaves| - 1) \times HorizontalMargin \leq PrintableArea_{width} $$
	\item garantiamo che un diagramma WBS rientri, per \emph{altezza}, nell'area di stampa se:
	$$ |levels| \times node_{height} + (|levels| - 1) \times VerticalMargin \leq PrintableArea_{height} $$
\end{itemize}
Intuitivamente questi limiti tutelano i casi peggiori, per esempio quando si
presenta un albero completo, il cui spazio occupato \`e il limite massimo espresso dalle formule.

\subsection{Metriche sullo spazio occupato dal diagramma Task Network}
Il diagramma task network pu\`o essere associato alla struttura dati grafo
aciclico (esprimiamo quindi le metriche solo per i casi \emph{well formed}). Come anche per il diagramma WBS, si vuole dare dei limiti sui parametri della task network entro i quali assicuriamo la stampa su carta nei margini fissati. I parametri che andremo ad utilizzare sono i seguenti:
\begin{itemize}
	\item \emph{larghezza dei nodi di inizio e fine} $start_{width}$, $end_{width}$: misurano la larghezza occupata dai nodi di inizio e fine,
	\item \emph{livelli} $levels$: l'insieme dei livelli. Ad ogni nodo viene associato un numero di livello e i nodi che hanno lo stesso numero si dicono appartenere allo stesso livello. L'etichettamento dei nodi col numero di livello inizia associando al nodo di inizio il valore zero e successivamente ad ogni nodo viene associato il numero di livello massimo tra quelli dei suoi padri incrementato di uno. Il numero di livelli $|levels|$ pu\`o quindi essere calcolato, dopo la fase di etichettamento, come il livello del nodo finale decrementato di uno,
	\item \emph{nodi di un livello fissato i} $level_i$:
	l'insieme di tutti i nodi appartenenti al livello $i$. La cardinalit\`a di questo insieme $|level_i|$ viene utilizzata per valutare che dimensioni pu\`o raggiungere in altezza il livello, e quindi il grafo,
	\item \emph{altezza e larghezza nodi} $node_{height}$, $node_{width}$: 
	analogo ai parametri dei diagrammi WBS,
	\item \emph{margine orizzontale tra nodi} $HorizontalMargin$ ($HM$): 
	analogo ai parametri dei diagrammi WBS,
	\item \emph{margine verticale tra nodi} $VerticalMargin$ ($VM$): 
	la distanza verticale minima tra i nodi, di default \`e fissata a \textbf{3mm} ma modificabile dalle opzioni di configurazione,
	\item \emph{archi che cambiano quota} $ChangingHeightArcs$ ($CHA$): 
	l'insieme di tutti gli archi che cambiano di quota, in quanto non riescono a raggiungere direttamente il nodo di arrivo. Il numero degli archi $|ChangingHeightArcs|$ che hanno questo comportamento \`e importante perch\'e ogni cambio di traiettoria viene fatto nello spazio tra due livelli adiacenti e non deve coincidere tra più archi al fine di evitare rappresentazioni ambigue. Il margine orizzontale tra due livelli aumenta quindi all'aumentare della cardinalit\`a di questo insieme,
	\item \emph{archi che attraversano un livello fissato i} $ArcsTroughtLevel_i$ ($ATL_i$):
	l'insieme di tutti gli archi che attraversano il livello $i$ per arrivare a un livello successivo o per entrare in una sotto attivit\`a di un task del livello $i$, come per i $ChangingHeightArcs$ il numero di questi archi $|ArcsTroughtLevel_i|$ \`e importante in quanto vanno ad aumentare l'altezza del livello che attraversano, e quindi la potenzale altezza dell'intero grafo,
	\item \emph{margini tra archi} $ArcMargin$ ($AM$) : indica la distanza minima tra due archi o tra un arco e un nodo, per default fissata a \textbf{2mm} ma modificabile dalel opzioni di configurazione,
	\item \emph{area stampabile} $PrintableArea_{width}$ ($PA_{width}$), $PrintableArea_{height}$ ($PA_{height}$): rappresentano le dimensioni dell'area stampabile del foglio, rispettivamente larghezza e altezza, quindi considerando gi\`a eliminati i margini di stampa.
\end{itemize}

Stabiliti i parametri passiamo a dichiarare le disequazioni rappresentanti i limiti entro i quali diamo garanzie:
\begin{itemize}
	\item garantiamo che un diagramma task network rientri, per \emph{larghezza}, nell'area di stampa se:
	$$ start_{width} + end_{width} + |levels| \times node_{width} + 	
	(|levels|+1) \times HM + |CHA| \times AM
	\leq PA_{width} $$
	\item garantiamo che un diagramma task network rientri, per \emph{altezza}, nell'area di stampa se:
	$$ max_{1 \leq i \leq |levels|}\{ |level_i| \times node_{height} +
	(|level_i|-1) \times VM + |ATL_i| \times AM \} 
	\leq PA_{height} $$
\end{itemize}




\part{Domain Model}

\chapter*{Release \textbf{1.1}}

\chapter{Overall diagram}

\begin{figure}[h!] 
	\centering
	\includegraphics[width=1\textwidth]{../Milestone1-DomainModel/DomainModel.png}
	\caption{Overall UML diagram}
	\label{fig:overallDiagram} 
\end{figure}

Questo diagramma comprende tutti i concetti che abbiamo identificato durante la
prima iterazione del blocco di analisi. 

Nella figura abbiamo una visione di insieme che pu\`o essere utile a fini di
codifica e progettazione del piano delle prove. Finch\`e si rimane invece nella
sfera della progettazione (analisi inclusa) potrebbe produrre dei dubbi in
quanto propone molti concetti; mentre si sta cercando di raffinare le varie relazioni
secondo noi \`e necessaria una vista pi\`u in dettaglio di composizioni
di pochi concetti che sono legati tra loro, lasciando tutti gli altri ad una
loro commento separato.

Procediamo nel seguito del documento nella descrizione di piccole composizioni
in modo da chiarire i motivi per cui sono stati creati concetti e relazioni fra
essi.

\chapter{Aspects descriptions}
% from here every document which is included with the input command must have a
% dedicated section whitin his body
\section{Task}
\label{sec:task}

\begin{figure}[h!] 
	\centering
	\includegraphics[width=0.4\textwidth]{../TaskDetail.png}
	\caption{task and its relations}
	\label{fig:task} 
\end{figure}

Molto probabilmente il concetto di \emph{Task} esiste gia nell'attuale
versione di \textbf{PMango 2.2.0}. Quello che abbiamo pensato \`e di introdurre
un \emph{glue layer} che ci permette di non apportare modifiche al codice
esistente di mango, ma lavorare con uno strato di intermezzo per essere il meno
intrusivi possibile e poter portare avanti il lavoro dipendendo solo dalle
nostri oggetti, facendo il minor riferimento al codice gia esistente.

Vogliamo rendere trasparente il concetto che un \emph{Task} sia un attivit\`a
singola (non scomponibile in sottoattivit\`a) che una attivit\`a scomposta. 

Costruiamo la relazione $\rightarrow$ che lega questi due concetti:
\begin{itemize}
  \item \emph{BasicTask} $\rightarrow$ attivit\`a di base, non ulteriormente
  scomponibili
  \item \emph{ComposedTask} $\rightarrow$ attivit\`a che sono composte da sotto
  attivit\`a
\end{itemize}
In questo modo possiamo trattare questi due tipi di attivit\`a in modo
interscambiabile e del tutto trasparente. Usando l'astrazione \emph{Task} non
ci importa se abbiamo una attivit\`a base o composta, in quanto cosi le abbiamo
portate ad avere interfaccie compatibili.

\section{TaskBox}
\label{sec:taskbox}

\begin{figure}[h!] 
	\centering
	\includegraphics[width=0.5\textwidth]{../TaskBoxDetail.png}
	\caption{taskbox and its specializations}
	\label{fig:taskbox} 
\end{figure}

Il \emph{TaskBox} \`e la rappresentazione grafica di un
\emph{Task} (\autoref{fig:task}). Questo concetto astrae su queste
specializzazioni:
\begin{itemize}
\item \emph{GanttTaskBox} che ci permetter\`a di costruire la rappresentazione
in un \emph{GanttChart} conformi alle norme fissate nel documento di specifica.

\item \emph{NodeTaskBox} che ci permetter\`a di costruire la rappresentazione
in un \emph{WBSChart} e in \emph{TaskNetworkChart} alle norme fissate nel 
documento di specifica.
\end{itemize}

Abbiamo usato il principio di incapsulare il concetto che varia, modellando il
concetto astratto di \emph{TaskBox} per avere questi vantaggi:
\begin{itemize}
  \item non legare un \emph{Chart} specifico a una rappresentazione specifica
  \item aggiungere una nuova rappresentazione consiste nel modellarla e
  dichiarare che si tratta di una specializzazione di \emph{TaskBox}
  \item potremo cambiare a runtime il tipo di rappresentazione voluta nel
  disegno di un \emph{Chart}, magari inserire in un \emph{WBSChart} una
  rappresentazione pensata per i \emph{GanttChart}
\end{itemize}
\section{Strip}
\label{sec:strip}

\begin{figure}[h!] 
	\centering
	\includegraphics[width=0.6\textwidth]{../StripDetail.png}
	\caption{kinds of strips}
	\label{fig:strip} 
\end{figure}
\section{Chart}
\label{sec:chart}

\begin{figure}[h!] 
	\centering
	\includegraphics[width=0.6\textwidth]{../ChartDetail.png}
	\caption{chart and building blocks}
	\label{fig:chart} 
\end{figure}
\section{Dependency}
\label{sec:dependency}

\begin{figure}[h!] 
	\centering
	\includegraphics[width=0.5\textwidth]{../Milestone1-DomainModel/DependencyDetail.png}
	\caption{dependencies}
	\label{fig:dependencies} 
\end{figure}

\emph{Dependency} modella il tipo di dipendenze che possiamo rappresentare in
un \emph{Chart}. Per implementare la specifica abbiamo bisogno di incapsulare
queste varianti:\footnote{dire in quali Chart vengono utilizzate}
\begin{itemize}
  \item \emph{Finish-ToStartDependency}: siano $a, b$ due \emph{Task} tali
  che $b$ non pu\`o iniziare finch\'e $a$ non sia completato. Questa relazione
  \`e catturata da questa specializzazione.
  \item \emph{HierarchycalDependency}: siano $a, b_{i}$ con $i= 1,\ldots,n \in
  N$, \emph{Task}s tali che $a$ \`e scoposto in $b_{i}$ \emph{Task}. Questa
  relazione \`e catturata da questa specializzazione.
\end{itemize}
\section{UserOption}
\label{sec:userOption}

\begin{figure}[h!] 
	\centering
	\includegraphics[width=0.7\textwidth]{../UserOptionDetail.png}
	\caption{UserOptions and choice}
	\label{fig:userOption} 
\end{figure}

Quando un generico client (potrebbe essere sia una persona fisica che un
oggetto astratto) della nostra implementazione della specifica vuole generare
un \emph{Chart} pu\`o guidare la generazione decidendo alcuni fattori che sono
di suo interesse. Questi fattori vengono modellati dal concetto di
\emph{UserOption}.

Ogni \emph{Chart} espone una lista di \emph{UserOption} per dare al client la
possibilit\`a di esprimere quali informazioni guidare. Questa lista varia da
\emph{Chart} a \emph{Chart}\footnote{creare un reference dove vengono mappate
questa relazione: potrebbe essere un appendici di questo documento??}.

Il concetto di lista di \emph{UserOption} \`e catturato in
\emph{UserOptionsChoice}.

Procediamo per passi: nelle prossime subsection osserviamo due aspetti che
trattarli insieme potrebbe non essere sufficiente per esporli in modo chiaro.

\subsection{\emph{UserOption}'s Instances}
Questa \`e stata una decisione non molto facile da prendere. Il problema \`e
questo: nella specifica abbiamo che per ogni \emph{Chart} il committente ha
dichiarato quali \emph{UserOption} mostrare. Queste per\`o non rappresentano un
concetto che vogliamo catturare nel nostro modello, ma allo stesso tempo sono
\emph{istanze} (un insieme discreto quindi) di elementi che fissa il
committente.

Per questo motivo decidiamo di codificare questo insieme discreto in questo
documento e la successiva enumerazione \`e da considerarsi parte integrante del
diagramma inserito come figura.

Rappresentiamo il concetto espresso sopra indicando due descrizioni con questa
struttura di codifica:
\begin{itemize}
  \item \emph{istanze}, dove inseriamo tutte le possibili \emph{UserOption} che
  non possono essere ancora raffinate
  \item \emph{specializzazioni} dove inseriamo tutte le possibili
  specializzazioni di \emph{UserOption} che possono essere ancora raffinate,
  ripetendo in modo ricorsivo questa struttura di codifica
\end{itemize}

Le successive descrizioni sono relative al concetto di \emph{UserOption}:
\begin{description}
  \item[istanze]\footnote{inserire qui la il mapping sui vari Chart?}
  \emph{WBSExplosionLevelUserOption, ActualTimeFrameOption, CompletitionBarOption, \\ PlannedDataOption, ActualDataOption,
  AlertMarkUserOption, ReplicateArrowUserOption, FindCriticalPathUserOption,
  WBSUserSpecificationUserLevel, \\WBSUserSpecificationUserLevel,
  ResourcesDetailsOption, TaskNameOption, CompleteDiagramUserOptions}
  \item[specializzazioni] \quad
  \begin{itemize}
    \item \emph{WBSTreeSpecification}
    \begin{description}
  \item[istanze] \emph{LevelSpecification, UserCustomSpecification}
  \item[specializzazioni] nessuna
\end{description}

\item \emph{TimeGrainUserOption}
    \begin{description}
  \item[istanze] \emph{WeaklyGrain, MonthlyGrain}
  \item[specializzazioni] nessuna
\end{description}

\item \emph{ImageDimensionUserOption}
    \begin{description}
  \item[istanze] \emph{CustomDim, FitInWindowDim, OptionalDim, DefaultDim}
  \item[specializzazioni] nessuna
\end{description}

\item \emph{TimeRangeUserOption}
    \begin{description}
  \item[istanze] \emph{CustomRange, WholeProjectRange, FromStartRange, ToEndRange}
  \item[specializzazioni] nessuna
\end{description}

  \end{itemize}
\end{description}

\subsection{\emph{UserOptionsChoice}}
Questo concetto \`e, secondo la nostra analisi, molto importante in quanto ci
permette di astrarre dal client che richiede una generazione.

Il motivo per cui abbiamo introdotto questo concetto \`e di poter lavorare lato
server usando \emph{UserOptionsChoice} per controllare quali informazioni il
client vuole guidare. In questo modo non siamo vincolati ad accedere ai dati
inviati per \emph{POST, GET} dalla form HTML, ma possiamo direttamente guardare
in \emph{UserOptionsChoice}. Queste ci permette di disaccopiare il processo di
generazione della maschera di input di una pagina HTML. 

Se vogliamo utilizzare il processo di generazione (che comunque \`e
server side) scrivendo un programma client (GUI o da riga di comando) che
costruisce una HTML request ad hoc (dovremo definire una grammatica e
attribuire la semantica ai contesti, questo \`e necessario, non ch\`e scrivere 
un parser), usando \emph{UserOptionsChoice} e il suo disaccoppiamento ci sar\`a
possibile farlo. 

Una volta ricevuta la response possiamo maneggiare la pagina
inviata come una response HTML valida e usarla per i nostri obiettivi (possiamo
rihiedere l'immagina generata, o il file PDF generato, salvandolo in
locale, oppure visualizzando lo stesso con un browser, ma possiamo anche
inserire in un db oppure farci dei test sopra\ldots).

Dovremo quindi costruire un oggetto che si incarica di costruire
\emph{UserOptionsChoice} in base al tipo di richiesta ricevuta (da una pagina
html come \`e il caso di PMango, oppure una richiesta da un client indipendente
scritto in un qualche linguaggio). Una volta costruito l'insieme delle
\emph{UserOption} \`e possibile iniziare la generazione. Questo sar\`a delegato
alla fase di progettazione.

\section{ReportSection}
\label{sec:reportSection}

\begin{figure}[h!] 
	\centering
	\includegraphics[width=0.5\textwidth]{../Milestone1-DomainModel/img/ReportSectionDetail.png}
	\caption{report section}
	\label{fig:reportSection} 
\end{figure}

Abbiamo da implemetare un requisito che vuole la possibilit\`a di aggiungere
alla reportistica un determinato \emph{Chart} con le relative \emph{UserOption}
scelte dall'utente. Modelliamo quindi il concetto di \emph{ReportSection} per
realizzare questo requisito. Come si vede dalla figura, \emph{ReportSection}
associa \emph{Chart} e \emph{UserOptionChoice}. Utilizziamo direttamente la
lista delle scelte\footnote{che viene costruita lato server} in modo da non
doverla costruire nella funzionalit\`a di assemblamento del report.



\part{Use cases}

\chapter*{Release \textbf{1.1}}

\chapter{Entire System UML diagram}
\begin{figure}[h!] \centering
\includegraphics[width=0.7\textwidth]{../Milestone2-UseCases/EntireSystem.png} 
\caption{Entire system UML diagram}
\label{fig:entireSystemDiagram}
\end{figure}

\chapter{Commons}
\begin{figure}[h!] \centering
\includegraphics[width=0.9\textwidth]{../Milestone2-UseCases/Commons/img/Overall.png}
\caption{Overall Commons UML diagram}
\label{fig:commonsOverallDiagram}
\end{figure}

\section{Make Dependencies}
\label{seq:makeDependencies}
\subsection{Basic course}
Il client prende un reference al GanttChartGenerator e domanda di creare la
rappresentazione delle dipendenze \footnote{decidere se rappresentare: \emph{il
Task dipende da} oppure \emph{il Task e' necessario per}} del \emph{Task}.

Il sistema esegue questi passi:
\begin{enumerate}
  \item recupera il \emph{Task}.
  \item se \emph{UserOptionChoice} contiene \emph{ShowDependenciesOption},
        allora rappresenta le dipendenze cdns. 
\end{enumerate}

\subsection{Alternative course}
\begin{description}
\item[not well formed project] Se la struttura al albero WBS del progetto non
\`e ben formata allora si deve cercare di dare la migliore euristica possibile
per la rappresentazione delle dipendenze.

\end{description}
\section{Make NodeTaskbox}
\label{seq:makeNodeTaskbox}
\subsection{Basic course}
Il client prende un reference al \emph{WBSChartGenerator} e domanda di creare la
rappresentazione grafica di un \emph{Task}.

Il sistema esegue questi passi:
\begin{enumerate}
  \item recupera il \emph{Task} da rappresentare.
  \item costruisce la rappresentazione grafica, il \emph{NodeTaskbox}
    in base alla scelta \emph{WBSTreeSpecification}. Il \emph{NodeTaskbox} \`e
    una composizione di \emph{Strip}. Il sistema costruisce una
    biezione tra \emph{BoxedStrip} e le scelte presenti in 
    \emph{UserOptionsChoice} in questo modo:
    \begin{itemize}
      % task name
      \item se \emph{UserOptionsChoice} contiene
      \emph{TaskNameOption}, allora costruisce una
      \emph{Strip} contenente il nome del \emph{Task} cdns.
      
      % resources
   	  \item se \emph{UserOptionsChoice} contiene \emph{ResourcesDetailsOption},
    	allora sul margine destro della \emph{GanttTaskBox} appendi la stringa
    	contenente la lista delle risorse cdns.
    	
    	Altrimenti appendi sul margine destro l'effort cdns.
      
      % planned time frame
      \item	se \emph{UserOptionsChoice} contiene \emph{PlannedTimeFrameOption} 
     allora costruisce due \emph{Strip} adiacenti contenenti rispettivamente le
     date di inizio e fine \emph{Task} cdns.
    	
	  % actual time frame
    	\item se \emph{UserOptionsChoice} contiene \emph{ActualTimeFrameOption}
    	allora costruisce due \emph{Strip} adiacenti contenenti rispettivamente 
    	le date di inizio e fine \emph{Task} reali cdns.
    	
		% planned data
    	\item se \emph{UserOptionsChoice} contiene \emph{PlannedDataOption} allora
    	costruisce tre \emph{Strip} adiacenti contenenti rispettivamente la durata
    	pianificata, lo sforzo complessivo pianificato, il costo
    	pianificato del \emph{Task} cdns.
    	
    	%actual data
    	\item se \emph{UserOptionsChoice} contiene \emph{ActualDataOption} allora
    	costruisce tre \emph{Strip} adiacenti contenenti rispettivamente la durata
    	“dall’inizio ad oggi”, lo sforzo complessivo “ad oggi” effettuato, il
    	costo complessivo “ad oggi” del \emph{Task} cdns.
    	
    	% completition bar
		\item se \emph{UserOptionsChoice} contiene \emph{CompletitionBarOption}
		allora costruisce la barra di completamento del \emph{Task} cdns.
		
    \end{itemize}
\end{enumerate}

\subsection{Alternative course}
\begin{description}
  \item[troncamento del nome del \emph{Task}] se la stringa scritta supera la
  dimensione fissata nel documento di specifica, allora il sistema la tronca
  cdns.


\end{description}

\section{Show Project Page}
\label{seq:showProjectPage}
\subsection{Basic course}
Il client digita l'URL della \emph{ProjectPage}\footnote{TODO:definire
ProjectPage nel documento dei Mockup}. \\
Il sistema invia in risposta la pagina richiesta aggiungendo all'insieme dei
tab presenti, tre tab relativi ai \emph{Chart} descritti nel documento
\textbf{Domain Model} e oggetto dell'appalto.\\
Il client fa click sul tab relativo al \emph{Chart} che vuole generare.\\
Il sistema invoca la specializzazione dello usecase \ref{seq:showUserOptions}
relativa al \emph{Chart} scelto, per costruire l'insieme delle possibili
\emph{UserOption}: dopo di che aggiunge questo insieme alla pagina di
risposta.\\
Il client effettua una azione, invocando una specializzazione di
\ref{seq:userAction}.
\section{Generate Chart}
\label{seq:generateChart}
Questo use case, come rappresentato nel diagramma UML
\autoref{fig:commonsOverallDiagram}, fornisce il punto di astrazione per altri
use case. Per questo motivo la descrizione del comportamento che  si
vuole modellare viene descritta in ogni specializzazione.
\section{Show UserOptions}
\label{seq:showUserOptions}
Questo use case, come rappresentato nel diagramma UML
\autoref{fig:commonsOverallDiagram}, fornisce il punto di astrazione per altri
use case. Usiamo questo formalismo per permettere ad ogni specializzazione di
esprimere solo quali sono le \emph{UserOption} disponibili per il relativo
\emph{Chart}.

\section{Make UserOptionsChoice}
\label{seq:makeUserOptionsChoice}
\subsection{Basic course}
Il sistema riceve una http request contenente una sequenza di \emph{UserOption}
che sono state selezionate dall'utente. Costruisce una
\emph{UserOptionsChoice} cosi: per ogni \emph{UserOption} indicata nella
request si aggiunge all'oggetto in costruzione.

Lato server abbiamo l'insieme di scelte disponibile per le azioni successive.
\section{User Action}
\label{seq:userAction}
Questo use case, come rappresentato nel diagramma UML
\autoref{fig:commonsOverallDiagram}, fornisce il punto di astrazione per altri
use case. Per questo motivo la descrizione del comportamento che  si
vuole modellare viene descritta in ogni specializzazione.
\section{Add to Report UserAction}
\label{seq:addToReportUserAction}
\subsection{Basic course}
Si assume che il client stia interagendo con il tab relativo al \emph{Chart}
che vuole generare.

Il client fa click sul pulsante ``Add to Report''.\\
Il sistema invoca lo use case \ref{seq:makeUserOptionsChoice} per costruirsi la
\emph{UserOptionsChoice}. \\
Il sistema aggiunge una \emph{ReportSection} per aggiungere alla reportistica
il \emph{Chart} richiesto, con la relativa \emph{UserOptionsChoice}. Queste
sezioni saranno elencate nella schermata della reportistica gi\`a esistente.
\section{Make PDF}
\label{seq:GanttMakePDF}
\subsection{Basic course}
Il client entra nella pagina relativa al diagramma Gantt e clicca sulla 
funzionalit\`a ''Make PDF''. Il sistema costruisce un oggetto in questo modo:
\begin{itemize}
  \item invoca lo use case \ref{seq:GanttMakeLeftColumn} per creare la
  colonna a sinistra.
  \item invoca lo use case per creare la gif 
\end{itemize}
Il sistema esegue il refresh della pagina aggiungendo accanto ai pulsanti di
reporting, una icona per permettere il download del file generato.  

\subsection{Alternative course}
\begin{description}
\item[nessuna]
\end{description}
\section{Refresh Chart}
\label{seq:refreshChart}
\subsection{Basic course}
Si assume che il client stia interagendo con il tab relativo al \emph{Chart}
che vuole generare.

Il client fa click sul pulsante ``Refresh''.\\
Il sistema esegue questi passi:
\begin{itemize}
  \item invoca lo use case \ref{seq:makeUserOptionsChoice} per costruirsi la
  \emph{UserOptionsChoice}.
  \item esegue una ricerca dei \emph{Task} che devono essere inclusi nel
  \emph{Chart}.  
  \item invoca la specializzazione di \ref{seq:generateChart} per la
  creazione del relativo \emph{Chart}
  \item invia al client una pagina di risposta inserendo la rappresentazione
  nel bottom della pagina.
\end{itemize}
\section{Select User Option}
\label{seq:selectUserOption}
\subsection{Basic course}
Si assume che il client stia interagendo con il tab relativo al \emph{Chart}
che vuole generare, quindi lo use case \ref{seq:showProjectPage} \`e gi\`a
stato completato. \\\\
Il client vuole selezionare alcune \emph{UserOption} per guidare la
rappresentazione delle informazioni che verranno codificate nel \emph{Chart}.\\
Queste sono rappresentate tramite una form html, sotto forma di controlli
grafici, dipendenti dal tipo e dalla semantica della \emph{UserOption} che
rappresentano. \footnote{inserire qua il mappaggio tra il tipo di UserOption e
il controllo grafico che viene bindato.}
Il sistema non effettua alcuna azione in quanto il fatto di memorizzare le
scelte fino al ``submit'' viene tenuto dalla form html stessa.

\section{Open in New Window}
\label{seq:openInNewWindow}
\subsection{Basic course}
Si assume che il client stia interagendo con il tab relativo al \emph{Chart}
che vuole generare.

Il client fa click sul pulsante ``Open In New Window''.\\
Il sistema esegue questi passi:
\begin{itemize}
  \item apre il collegamento in una nuova pagina vuota
  \item invoca lo use case \ref{seq:refreshChart} per costruire il \emph{Chart}
  \item visualizza il \emph{Chart} nella nuova pagina
\end{itemize}

\chapter{Gantt chart}
\section*{Overall UML diagram}
\begin{figure}[h!] \centering
\includegraphics[width=0.8\textwidth]{../Milestone2-UseCases/Gantt/img/GanttChart.png}
\caption{Gantt Overall UML diagram}
\label{fig:ganttDiagram}
\end{figure}

\section{Make Left Column}
\label{seq:GanttMakeLeftColumn}
\subsection{Basic course}
Il client prende un reference al GanttChartGenerator e domanda la funzionalit\'a
per creare la colonna di sinistra del diagramma. \\
Il sistema costruisce una colonna di larghezza di dimensione \emph{fissa}.
Guarda se in UserOptionChoises \`e presente l'opzione TaskNameOption:
\begin{itemize}
  \item se \'e presente allora calcola la larghezza
\begin{displaymath}
	width_{nor}=lenght(max\lbrace id | \forall task: id =
WBS\_identifier(task) \rbrace) 
\end{displaymath}
dove la funzione $WBS\_identifier$ restituisce il WBS identifier relativo al 
parametro $task$ del dominio.

Poi $\forall task$ scrive $WBS\_identifier(task)$, un task per ogni riga.

\item altrimenti calcola
\begin{eqnarray}
width_{opt}=lenght(max\lbrace ::(id, task\_name) | \forall task: \\ id =
WBS\_identifier(task), \\ task\_name = Get\_TaskName(task)
\rbrace)
\end{eqnarray}
dove l'operatore $::$ concatena le stringhe passate come parametro, e
l'operatore \\$Get\_TaskName$ restituisce il nome del parametro
task.

Poi $\forall task$ scrive $::(WBS\_identifier(task), Get\_TaskName(task)$, un
task per ogni riga.
\end{itemize}

\subsection{Alternative course}
\begin{description}
\item[troncamento dei TaskName] se $width_{opt} >
\frac{width_{diagram}}{6}$, con $width_{diagram}$ uguale alla larghezza totale
del diagramma, allora sia 
\begin{equation}cutted = cut(::(WBS\_identifier(task),
Get\_TaskName(task))\end{equation} 
la funzione $cut$ cancella caratteri in coda alla stringa in ingresso e
restituisce $cutted$, contenente i caratteri non cancellati, in modo che $cutted
= \frac{width_{diagram}}{6}-3$. \\ Il sistema scrive 
\begin{equation} 
::(cut(::(WBS\_identifier(task), Get\_TaskName(task)), ''\ldots'')
\end{equation}
\item[troncamento del WBS identifier] se avessi un identifier troppo lungo
dovrei fare lo stesso??
\end{description}
\section{Make GanttTaskbox}
\label{seq:makeGanttTaskbox}
\subsection{Basic course}
Il client prende un reference al GanttChartGenerator e domanda di creare la
rappresentazione grafica di un \emph{Task}. 

Il sistema esegue questi passi:
\begin{enumerate}
  \item recupera il \emph{Task} da rappresentare
  \item costruisce la rappresentazione grafica, il \emph{GanttTaskBox}
    in base alla scelta \emph{WBSTreeSpecification}.
  \item \`E possibile codificare delle informazioni aggiuntive:
    \begin{itemize}
   	  \item se \emph{UserOptionChoice} contiene \emph{ResourcesDetailsOption},
    	allora sul margine destro della \emph{GanttTaskBox} appendi la stringa
    	contenente la lista delle risorse cdns.
    	
    	Altrimenti appendi sul margine destro l'effort cdns.
    \end{itemize}
    
\end{enumerate}

\subsection{Alternative course}
\begin{description}
\item[troncamento delle resourses] se le informazioni testuali aggiuntive
superano la dimensione definita nel documento di specifica, allora si troncano
cdns.

\end{description}
\section{Make Right Column}
\label{seq:GanttRightRepresentation}
\subsection{Basic course}
Il client prende un reference al GanttChartGenerator e domanda la funzionalit\'a
per creare la colonna di destra del diagramma. \\
Il sistema costruisce una colonna di larghezza di dimensione \emph{fissa} per 
riempire lo spazio a destra restante del diagramma (per far posto alla colonna
di sinistra). La generazione dell'immagine \`e guidata dall'appartenenza delle
seguenti voci a UserOptionChoises:
\begin{description}
\item[ResourcesDetailOption] \quad
\begin{itemize}
  \item se \emph{non} \'e presente allora inserisci scrive le informazioni
  sullo sforzo seguendo questo pattern \textbf{actual/planned effort}
  \item altrimenti scrive le informazioni detailed for each resource assigned to
  the task
\end{itemize}
\end{description}

\subsection{Alternative course}
\begin{description}
\item[troncamento delle resourses] if the resourses string exceeds $\frac{1}{6}$
of the whole diagram width then resource names are truncated and marked by ''...'', if
needed the list is truncated too.

\end{description}
\section{Make PDF}
\label{seq:GanttMakePDF}
\subsection{Basic course}
Il client entra nella pagina relativa al diagramma Gantt e clicca sulla 
funzionalit\`a ''Make PDF''. Il sistema costruisce un oggetto in questo modo:
\begin{itemize}
  \item invoca lo use case \ref{seq:GanttMakeLeftColumn} per creare la
  colonna a sinistra.
  \item invoca lo use case per creare la gif 
\end{itemize}
Il sistema esegue il refresh della pagina aggiungendo accanto ai pulsanti di
reporting, una icona per permettere il download del file generato.  

\subsection{Alternative course}
\begin{description}
\item[nessuna]
\end{description}
\section{Gantt User Options}
\label{seq:ganttUserOption}
\subsection{Basic course}
Questo use case estende (specializza) lo use case \ref{seq:showUserOptions}.
\\ \\
Il sistema costruisce le \emph{UserOption} relative al \emph{GanttChart} in un
unico passo, composto da queste azioni omogenee:
\begin{itemize}
  \item rappresenta \textbf{TaskNameUserOption} con una checkbox e associa una
  label descrittiva
  \item rappresenta \textbf{EffortInformationUserOption} con una checkbox e associa una
  label descrittiva
  \item rappresenta \textbf{FinishToStartDependenciesUserOption} con una checkbox e associa una
  label descrittiva
  \item rappresenta \textbf{ReplicateArrowUserOption} con una checkbox e associa una
  label descrittiva
  \item rappresenta \textbf{UseDifferentPatternForCrossingLinesUserOption} con una checkbox e associa una
  label descrittiva
  \item rappresenta le istanze di \textbf{WBSTreeSpecification} come voci di una
  combobox. Quando viene scelta la voce \textbf{LevelSpecificationUserOption}, allora
  compare vicino alla combobox, un'altra combobox con i livelli disponibili
  (come quella che esiste nell'attuale versione di PMango)
  \item rappresenta le istanze di \textbf{TimeGrainUserOption} come voci di una
  combobox.
  \item rappresenta le istanze di \textbf{TimeRangeUserOption} come voci di una
  combobox. Si aggiungo ulteriori controlli in base alle seguenti scelte:
  \begin{description}
    \item[CustomRangeUserOption] allora costruisci accanto alla combo, due
    calendari per selezionare la coppia di date.
    \item[FromStartRangeUserOption] allora costruisci accanto alla combo, un
    calendario per selezionare la data di inizio.
    \item[ToEndRangeUserOption] allora costruisci accanto alla combo, un
    calendario per selezionare la data di fine.
  \end{description}  
  \item rappresenta le istanze di \textbf{ImageDimensionUserOption} come voci di
  una combobox. Si aggiungo ulteriori controlli in base alle seguenti scelte:
  \begin{description}
    \item[CustomDimUserOption] allora costruisci accanto alla combo, due
    textbox per specificare la coppia di interi rappresentate le dimensioni del
    \emph{Chart} desiderate.
  \end{description}
  \item rappresenta \textbf{OpenInNewWindowUserOption} con una checkbox e associa una
  label descrittiva
\end{itemize}

\chapter{WBS chart}
\section*{Overall UML diagram}
\begin{figure}[h!] \centering
\includegraphics[width=0.8\textwidth]{../Milestone2-UseCases/WBS/WBSChart.png}
\caption{WBS UML diagram}
\label{fig:WBSdiagram}
\end{figure}

\section{Make Hierarchycal Dependencies}
\label{seq:makeHierarchycalDependencies}
\subsection{Basic course}
Il client richiede la funzionalità di rappresentazione della relazione 
gerarchica di un task.

Il sistema esegue questi passi:
\begin{enumerate}
  \item in base alle informazioni in \emph{WBSTreeSpecification}, recupera i
  figli del task (non l’intera discendenza, solo quelli di livello successivo) e
  li dispone graficamente al di sotto di esso.
  \item La distanze fra livelli e tra taskbox appartenenti allo stesso
  livello rispettano queste regole:
  \begin{itemize}
    \item la distanza \emph{inter-Taskbox} fra coppie di
    \emph{Taskbox} adiacenti, rappresentati sullo stesso livello, deve essere
    equa ed uguale per ogni coppia (questa regola si ripete in modo iterativo
    per ogni coppia appartenente ad un livello $l$)
	\item la distanza \emph{inter-Level} fra coppie di livelli contenenti le 
	rappresentazioni dei task padre e i suoi figli deve essere equa e uguale
	per ogni coppia di livelli (questa regola si applica ricorsivamente al livello
	dei figli, nel caso che almeno un figlio sia composto da subtask)
  \end{itemize} 
  
  \item Il sistema collega il task ai suoi figli con linee spezzate cdns.
\end{enumerate}

\subsection{Alternative course}
\begin{description}
\item[WBSExplosionLevel = 0] se il livello di visualizzazione della WBSStructure
richiesto si limita alla root, crea il solo un \emph{NodeTaskbox} rappresentante
l’intero progetto.
\end{description}


\section{Make PDF}
\label{seq:GanttMakePDF}
\subsection{Basic course}
Il client entra nella pagina relativa al diagramma Gantt e clicca sulla 
funzionalit\`a ''Make PDF''. Il sistema costruisce un oggetto in questo modo:
\begin{itemize}
  \item invoca lo use case \ref{seq:GanttMakeLeftColumn} per creare la
  colonna a sinistra.
  \item invoca lo use case per creare la gif 
\end{itemize}
Il sistema esegue il refresh della pagina aggiungendo accanto ai pulsanti di
reporting, una icona per permettere il download del file generato.  

\subsection{Alternative course}
\begin{description}
\item[nessuna]
\end{description}
\section{WBS User Options}
\label{seq:wbsUserOption}
\subsection{Basic course}
Questo use case estende (specializza) lo use case \ref{seq:showUserOptions}.
\\ \\
Il sistema costruisce le \emph{UserOption} relative al \emph{WBSChart} in un
unico passo, composto da queste azioni omogenee:
\begin{itemize}
  \item rappresenta \textbf{TaskNameUserOption} con una checkbox e associa una
  label descrittiva
  \item rappresenta \textbf{PlannedDataUserOption} con una checkbox e associa una
  label descrittiva
  \item rappresenta \textbf{PlannedTimeFrameUserOption} con una checkbox e associa una
  label descrittiva
  \item rappresenta \textbf{ResourcesUserOption} con una checkbox e associa una
  label descrittiva
  \item rappresenta \textbf{ActualTimeFrameUserOption} con una checkbox e associa una
  label descrittiva
  \item rappresenta \textbf{ActualDataUserOption} con una checkbox e associa una
  label descrittiva
  \item rappresenta \textbf{AlertMarkUserOption} con una checkbox e associa una
  label descrittiva
  \item rappresenta le istanze di \textbf{WBSTreeSpecification} come voci di una
  combobox. Quando viene scelta la voce \textbf{LevelSpecificationUserOption}, allora
  compare vicino alla combobox, un'altra combobox con i livelli disponibili
  (come quella che esiste nell'attuale versione di PMango)
  \item rappresenta le istanze di \textbf{ImageDimensionUserOption} come voci di
  una combobox. Si aggiungo ulteriori controlli in base alle seguenti scelte:
  \begin{description}
    \item[CustomDimUserOption] allora costruisci accanto alla combo, due
    textbox per specificare la coppia di interi rappresentate le dimensioni del
    \emph{Chart} desiderate.
  \end{description}
  \item rappresenta \textbf{OpenInNewWindowUserOption} con una checkbox e associa una
  label descrittiva
\end{itemize}

\chapter{TaskNetwork chart}
\section*{Overall UML diagram}
\begin{figure}[h!] \centering
\includegraphics[width=0.8\textwidth]{../Milestone2-UseCases/TaskNetwork/img/TNChart.png}
\caption{TaskNetwork UML diagram}
\label{fig:TNdiagram}
\end{figure}

\section{Generate TaskNetwork Chart}
\label{seq:generateTNChart}

\subsection{Basic course}
Il client richiede di generare un \emph{TaskNetworkChart}.\\
Il sistema costruisce un oggetto in questo modo:
\begin{itemize}
   \item invoca lo use case \ref{seq:makeNodeTaskbox} per creare i
  \emph{NodeTaskBox} del diagramma.
  \item si possono aggiungere le seguenti informazioni:
  \begin{itemize}
    \item se \emph{UserOptionsChoice} contiene \emph{ShowCriticalPath} allora
    	calcola il critical path rispetto al grafo rappresentato
    \item se \emph{UserOptionsChoice} contiene \emph{ShowCriticalPathTable}
    allora invoca lo use case \ref{seq:createCriticalPathTable}
 	\item se \emph{UserOptionsChoice} contiene \emph{ShowDependencies} allora per
 	 ogni \emph{TaskBox} rappresentata, invoca lo use case \ref{seq:makeDependencies}
  \end{itemize}
\end{itemize}
\section{Create Critical Path Table}
\label{seq:createCriticalPathTable}

\subsection{Basic course}
Si assume che i critical path siano gi\`a stati calcolati e siano gi\`a
rappresentati secondo le opzioni presenti in \emph{UserOptionsChoice}.\\

Il client richiede di generare la tabella riassuntiva per i \emph{critical
path}.\\ 
Il sistema costruisce una tabella definita cosi:
\begin{table}[h!]
  \begin{center}
    \begin{tabular}{| l | l | l | l |}
    \hline
    \textbf{durata} & \textbf{effort} & \textbf{ultimo gap} &
    \textbf{visualizzato} \\
	\hline \ldots & \ldots & \ldots & *\\
    \hline
    \end{tabular}
  \end{center}
\end{table}
Ogni critical path calcolato rappresenta una entry nella tabella.

\section{TaskNetwork User Options}
\label{seq:tNUserOption}
\subsection{Basic course}
Questo use case estende (specializza) lo use case \ref{seq:showUserOptions}.
\\ \\
Il sistema costruisce le \emph{UserOption} relative al \emph{TaskNetworkChart}
in un unico passo, composto da queste azioni omogenee:
\begin{itemize}
  \item rappresenta \textbf{TaskNameUserOption} con una checkbox e associa una
  label descrittiva
  \item rappresenta \textbf{PlannedDataUserOption} con una checkbox e associa una
  label descrittiva
  \item rappresenta \textbf{PlannedTimeFrameUserOption} con una checkbox e associa una
  label descrittiva
  \item rappresenta \textbf{ResourcesUserOption} con una checkbox e associa una
  label descrittiva
  \item rappresenta \textbf{ActualTimeFrameUserOption} con una checkbox e associa una
  label descrittiva
  \item rappresenta \textbf{ActualDataUserOption} con una checkbox e associa una
  label descrittiva
  \item rappresenta \textbf{AlertMarkUserOption} con una checkbox e associa una
  label descrittiva
  \item rappresenta \textbf{ReplicateArrowUserOption} con una checkbox e associa una
  label descrittiva
  \item rappresenta \textbf{UseDifferentPatternForCrossingLinesUserOption} con una checkbox e associa una
  label descrittiva
  \item rappresenta \textbf{TimeGapsUserOption} con una checkbox e associa una
  label descrittiva
  \item rappresenta \textbf{ShowCompleteDiagramDependencies} con una checkbox e associa una
  label descrittiva
  \item rappresenta le istanze di \textbf{WBSTreeSpecification} come voci di una
  combobox. Quando viene scelta la voce \textbf{LevelSpecificationUserOption}, allora
  compare vicino alla combobox, un'altra combobox con i livelli disponibili
  (come quella che esiste nell'attuale versione di PMango)
  \item rappresenta \textbf{CriticalPathUserOption} con una checkbox e associa una
  label descrittiva
  \begin{description}
    \item[MaxCriticalPathNumberUserOption] se la
    \emph{CriticalPathUserOption} padre viene selezionata, allora
    costruire vicino ad essa un textbox per permettere di specificare l'intero.
  \end{description}
  \item rappresenta le istanze di \textbf{ImageDimensionUserOption} come voci di
  una combobox. Si aggiungo ulteriori controlli in base alle seguenti scelte:
  \begin{description}
    \item[CustomDimUserOption] allora costruisci accanto alla combo, due
    textbox per specificare la coppia di interi rappresentate le dimensioni del
    \emph{Chart} desiderate.
  \end{description}
  \item rappresenta \textbf{OpenInNewWindowUserOption} con una checkbox e associa una
  label descrittiva
\end{itemize}

\part{Mockups}

\chapter*{Release \textbf{1.0}}

\chapter{Gantt chart}

\begin{sidewaysfigure}[h!] 
\centering
\includegraphics[width=1\textwidth]{../Mockup/Gantt.png}
\caption{Gantt tab mockup}
\end{sidewaysfigure}

\chapter{WBS chart}

\begin{sidewaysfigure}[h!] 
\centering
\includegraphics[width=1\textwidth]{../Mockup/WBS.png}
\caption{WBS tab mockup}
\end{sidewaysfigure}

\chapter{Task Network chart}

\begin{sidewaysfigure}[h!] 
\centering
\includegraphics[width=1\textwidth]{../Mockup/TN.png}
\caption{Task Network tab mockup}
\end{sidewaysfigure}

\end{document}