\section{Make GanttTaskbox}
\label{seq:makeGanttTaskbox}
\subsection{Basic course}
Il client prende un reference al GanttChartGenerator e domanda di creare la
rappresentazione grafica di un \emph{Task}. 

Il sistema esegue questi passi:
\begin{enumerate}
  \item recupera il \emph{Task} da rappresentare
  \item costruisce la rappresentazione grafica, il \emph{GanttTaskBox}
    in base alla scelta \emph{WBSTreeSpecification}.
  \item \`E possibile codificare delle informazioni aggiuntive:
    \begin{itemize}
   	  \item se \emph{UserOptionChoice} contiene \emph{ResourcesDetailsOption},
    	allora sul margine destro della \emph{GanttTaskBox} appendi la stringa
    	contenente la lista delle risorse cdns.
    	
    	Altrimenti appendi sul margine destro l'effort cdns.
    \end{itemize}
  \item invoca lo use case \ref{seq:makeDependencies}
    
\end{enumerate}

\subsection{Alternative course}
\begin{description}
\item[troncamento delle resourses] se le informazioni testuali aggiuntive
superano la dimensione definita nel documento di specifica, allora si troncano
cdns.

\end{description}