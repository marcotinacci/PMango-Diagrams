\section{Make Left Column}
\label{seq:GanttMakeLeftColumn}
\subsection{Basic course}
Il client prende un reference al GanttChartGenerator e domanda la funzionalit\'a
per creare la colonna di sinistra del diagramma. \\
Il sistema costruisce una colonna di larghezza di dimensione \emph{fissa}.
Guarda se in UserOptionChoises \`e presente l'opzione TaskNameOption:
\begin{itemize}
  \item se \'e presente allora calcola la larghezza
\begin{displaymath}
	width_{nor}=lenght(max\lbrace id | \forall task: id =
WBS\_identifier(task) \rbrace) 
\end{displaymath}
dove la funzione $WBS\_identifier$ restituisce il WBS identifier relativo al 
parametro $task$ del dominio.

Poi $\forall task$ scrive $WBS\_identifier(task)$, un task per ogni riga.

\item altrimenti calcola
\begin{eqnarray}
width_{opt}=lenght(max\lbrace ::(id, task\_name) | \forall task: \\ id =
WBS\_identifier(task), \\ task\_name = Get\_TaskName(task)
\rbrace)
\end{eqnarray}
dove l'operatore $::$ concatena le stringhe passate come parametro, e
l'operatore \\$Get\_TaskName$ restituisce il nome del parametro
task.

Poi $\forall task$ scrive $::(WBS\_identifier(task), Get\_TaskName(task)$, un
task per ogni riga.
\end{itemize}

\subsection{Alternative course}
\begin{description}
\item[troncamento dei TaskName] se $width_{opt} >
\frac{width_{diagram}}{6}$, con $width_{diagram}$ uguale alla larghezza totale
del diagramma, allora sia 
\begin{equation}cutted = cut(::(WBS\_identifier(task),
Get\_TaskName(task))\end{equation} 
la funzione $cut$ cancella caratteri in coda alla stringa in ingresso e
restituisce $cutted$, contenente i caratteri non cancellati, in modo che $cutted
= \frac{width_{diagram}}{6}-3$. \\ Il sistema scrive 
\begin{equation} 
::(cut(::(WBS\_identifier(task), Get\_TaskName(task)), ''\ldots'')
\end{equation}
\item[troncamento del WBS identifier] se avessi un identifier troppo lungo
dovrei fare lo stesso??
\end{description}