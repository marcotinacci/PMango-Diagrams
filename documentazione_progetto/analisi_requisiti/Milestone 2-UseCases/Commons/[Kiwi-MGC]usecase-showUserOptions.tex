\section{Show UserOptions}
\label{seq:showUserOptions}
\subsection{Basic course}
Il client digita l'URL della \emph{ProjectPage}\footnote{TODO:definire
ProjectPage nel documento dei Mockup}. 

Il sistema invia in risposta la pagina richiesta con i tre tab relativi ai
\emph{Chart} descritti nel documento \textbf{Domain Model}. 

Il client fa click sul tab relativo al \emph{Chart} che vuole visualizzare.

Il sistema costruisce il contenuto del tab inserendo una sequenza di
\emph{UserOption}. 

Il client esegue la sua selezione delle informazioni che
vuole visualizzare nel \emph{Chart} e clicca su un oggetto\footnote{come posso
speficare questo oggetto?} per domandare la generazione del \emph{Chart}.

Il sistema esegue questi passi:
\begin{enumerate}
  \item costruisce \emph{UserOptionsChoice} in base alle opzioni selezionate
  dal client.
  \item recupera i \emph{Task} da includere nel \emph{Chart}.
  \item esegue la generazione\footnote{dire che la generazione astrae sui
  formati: la richiesta potrebbe essere o la visualizzazione della gif nel
  browser come attualmente fa mango, altrimenti la richiesta di generazione
  pdf. Inserire questo paragrafo come un description sotto il punto a cui
  appartiene questa nota.}
\end{enumerate}

\subsection{Alternative course}
\begin{description}
\item[Nessuna \emph{UserOption} selezionata] Il client invia un request con 
selezione delle \emph{UserOption} vuota. Quindi il sistema genera il
\emph{Chart} con delle \emph{UserOption} di default\footnote{dedicare un
paragrafo in un qualche documento per definire quali sono le opzioni di
default per ciascun chart.}.
\end{description}