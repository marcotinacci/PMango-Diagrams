\documentclass[a4paper, 12pt]{article}

\usepackage{charter}
\usepackage{makeidx}
\usepackage{fancyhdr}
\usepackage{hyperref}
\usepackage[utf8]{inputenc}
\usepackage{graphicx}
\usepackage[left=2cm, right=2cm]{geometry}
\usepackage{latexsym}
\usepackage{amsmath, amsthm, amssymb}
\usepackage{rotating}


\begin{titlepage}
\title{Domain Model}
\author{Release 0.4}
\date{\today \\Firenze \\\begin{figure}[h] \centering
\includegraphics[width=0.2\textwidth]{../../../images/logokiwi.png} 
\end{figure}
}
\end{titlepage}

\pagestyle{fancy}

\begin{document}

\maketitle

\newpage

\section*{Approvazione, redazione, lista distribuzione}
\begin{table}[h!]
  \begin{center}
    \begin{tabular}{| l | l | p{60mm} |}
    \hline
    \textbf{approvato da} & \textbf{il giorno} & \textbf{firma} \\
	\hline    
	Marco Tinacci &  &  \\
    \hline
    \end{tabular}
  \end{center}
\end{table}

\begin{table}[h!]
  \begin{center}
    \begin{tabular}{| l | l | p{60mm} |}
    \hline
    \textbf{redatto da} & \textbf{il giorno} & \textbf{firma} \\
    \hline
    Francesco Calabri &  &  \\
    \hline
	Manuele Paulantonio &  &  \\
    \hline    
	Massimo Nocentini &  &  \\
    \hline
    \end{tabular}
  \end{center}
\end{table}

\begin{table}[h!]
  \begin{center}
    \begin{tabular}{| l | l | p{60mm} |}
    \hline
    \textbf{distribuito a} & \textbf{il giorno} & \textbf{firma} \\
	\hline    
	Daniele Poggi &  &  \\
    \hline
	Niccol\'o Rogai &  &  \\
    \hline
	Marco Tinacci &  &  \\
    \hline
    \end{tabular}
  \end{center}
\end{table}

\newpage

\tableofcontents

\newpage

\section*{Introduzione}

\newpage

\section{Overall diagram}
\begin{figure}[h!] 
	\centering
	\includegraphics[width=1\textwidth]{../DomainModel.png}
	\caption{Overall UML diagram}
	\label{fig:overallDiagram} 
\end{figure}

Questo diagramma comprende tutti i concetti che abbiamo identificato durante la
prima iterazione del blocco di analisi. 

Nella figura abbiamo una visione di insieme che pu\`o essere utile a fini di
codifica e progettazione del piano delle prove. Finch\`e si rimane invece nella
sfera della progettazione (analisi inclusa) potrebbe produrre dei dubbi in
quanto propone molti concetti; mentre si sta cercando di raffinare le varie relazioni
secondo noi \`e necessaria una vista pi\`u in dettaglio di composizioni
di pochi concetti che sono legati tra loro, lasciando tutti gli altri ad una
loro commento separato.

Procediamo nel seguito del documento nella descrizione di piccole composizioni
in modo da chiarire i motivi per cui sono stati creati concetti e relazioni fra
essi.

% from here every document which is included with the input command must have a
% dedicated section whitin his body
\section{Task}
\label{sec:task}

\begin{figure}[h!] 
	\centering
	\includegraphics[width=0.4\textwidth]{../TaskDetail.png}
	\caption{task and its relations}
	\label{fig:task} 
\end{figure}

Molto probabilmente il concetto di \emph{Task} esiste gia nell'attuale
versione di \textbf{PMango 2.2.0}. Quello che abbiamo pensato \`e di introdurre
un \emph{glue layer} che ci permette di non apportare modifiche al codice
esistente di mango, ma lavorare con uno strato di intermezzo per essere il meno
intrusivi possibile e poter portare avanti il lavoro dipendendo solo dalle
nostri oggetti, facendo il minor riferimento al codice gia esistente.

Vogliamo rendere trasparente il concetto che un \emph{Task} sia un attivit\`a
singola (non scomponibile in sottoattivit\`a) che una attivit\`a scomposta. 

Costruiamo la relazione $\rightarrow$ che lega questi due concetti:
\begin{itemize}
  \item \emph{BasicTask} $\rightarrow$ attivit\`a di base, non ulteriormente
  scomponibili
  \item \emph{ComposedTask} $\rightarrow$ attivit\`a che sono composte da sotto
  attivit\`a
\end{itemize}
In questo modo possiamo trattare questi due tipi di attivit\`a in modo
interscambiabile e del tutto trasparente. Usando l'astrazione \emph{Task} non
ci importa se abbiamo una attivit\`a base o composta, in quanto cosi le abbiamo
portate ad avere interfaccie compatibili.

\section{TaskBox}
\label{sec:taskbox}

\begin{figure}[h!] 
	\centering
	\includegraphics[width=0.5\textwidth]{../TaskBoxDetail.png}
	\caption{taskbox and its specializations}
	\label{fig:taskbox} 
\end{figure}

Il \emph{TaskBox} \`e la rappresentazione grafica di un
\emph{Task} (\autoref{fig:task}). Questo concetto astrae su queste
specializzazioni:
\begin{itemize}
\item \emph{GanttTaskBox} che ci permetter\`a di costruire la rappresentazione
in un \emph{GanttChart} conformi alle norme fissate nel documento di specifica.

\item \emph{NodeTaskBox} che ci permetter\`a di costruire la rappresentazione
in un \emph{WBSChart} e in \emph{TaskNetworkChart} alle norme fissate nel 
documento di specifica.
\end{itemize}

Abbiamo usato il principio di incapsulare il concetto che varia, modellando il
concetto astratto di \emph{TaskBox} per avere questi vantaggi:
\begin{itemize}
  \item non legare un \emph{Chart} specifico a una rappresentazione specifica
  \item aggiungere una nuova rappresentazione consiste nel modellarla e
  dichiarare che si tratta di una specializzazione di \emph{TaskBox}
  \item potremo cambiare a runtime il tipo di rappresentazione voluta nel
  disegno di un \emph{Chart}, magari inserire in un \emph{WBSChart} una
  rappresentazione pensata per i \emph{GanttChart}
\end{itemize}
\section{Strip}
\label{sec:strip}

\begin{figure}[h!] 
	\centering
	\includegraphics[width=0.6\textwidth]{../StripDetail.png}
	\caption{kinds of strips}
	\label{fig:strip} 
\end{figure}
\section{Chart}
\label{sec:chart}

\begin{figure}[h!] 
	\centering
	\includegraphics[width=0.6\textwidth]{../ChartDetail.png}
	\caption{chart and building blocks}
	\label{fig:chart} 
\end{figure}
\end{document}